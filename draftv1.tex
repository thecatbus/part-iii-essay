\documentclass[a4paper]{article}
\usepackage[hmargin={30mm,30mm},vmargin={30mm,30mm}]{geometry}


% FONT
\usepackage[T1]{fontenc}
\usepackage{charter}
%\usepackage{sectsty}
%\allsectionsfont{\scshape}


% CITATION STYLE
\usepackage{harvard}


% MATHS
\usepackage{euler}
\usepackage{amsmath, amsthm, amssymb, mathtools, stmaryrd}
\usepackage{tikz-cd}
\usepackage{enumerate}
%\usepackage{unicode-math}


% LINE SPACING
\usepackage[parfill]{parskip}
\usepackage[capitalise]{cleveref}
\renewcommand{\baselinestretch}{1.25}


% TODOS
\usepackage[colorinlistoftodos]{todonotes}


% TABLE OF CONTENTS
\usepackage{tocloft}
\renewcommand\contentsname{}
\renewcommand\cftsecfont{}
\renewcommand\cftsecpagefont{}
\setlength\cftbeforesecskip{3pt}
\setcounter{tocdepth}{2}


% THEOREM STYLES
\theoremstyle{definition}
\newtheorem{defn}{Definition}[section]

\newtheorem{theorem}[defn]{Theorem}
\newtheorem*{theorem*}{Theorem}
\newtheorem{prop}[defn]{Proposition}
\newtheorem{lemma}[defn]{Lemma}
\newtheorem{cor}[defn]{Corollary}

\newtheorem{example}[defn]{Example}

\theoremstyle{remark}
\newtheorem{remark}[defn]{Remark}


% NEW ARROWS
\usepackage{stackrel}
\newcommand{\leftrarrows}{\mathrel{\raise.75ex\hbox{\oalign{%
  $\scriptstyle\leftarrow$\cr
  \vrule width0pt height.5ex$\hfil\scriptstyle\relbar$\cr}}}}
\newcommand{\lrightarrows}{\mathrel{\raise.75ex\hbox{\oalign{%
  $\scriptstyle\relbar$\hfil\cr
  $\scriptstyle\vrule width0pt height.5ex\smash\rightarrow$\cr}}}}
\newcommand{\Rrelbar}{\mathrel{\raise.75ex\hbox{\oalign{%
  $\scriptstyle\relbar$\cr
  \vrule width0pt height.5ex$\scriptstyle\relbar$}}}}
\newcommand{\longleftrightarrows}{\leftrarrows\joinrel\Rrelbar\joinrel\lrightarrows}

\makeatletter
\def\leftrightarrowsfill@{\arrowfill@\leftrarrows\Rrelbar\lrightarrows}
\newcommand{\xleftrightarrows}[2][]{\ext@arrow 3399\leftrightarrowsfill@{#1}{#2}}
\makeatother

\usepackage{scalerel}
\newcommand{\simrightarrow}{\mathrel{\ooalign{
     $\to$\cr
     \hidewidth\raise.3em\hbox{$\scaleobj{.7}{\sim}\mkern7mu$}\cr
    }
  }
}


% CUSTOM COMMANDS
\newcommand{\Exter}{\mathchoice{{\textstyle\bigwedge}}%
    {{\bigwedge}}%
    {{\textstyle\wedge}}%
    {{\scriptstyle\wedge}}}

\newcommand{\grMod}{\ensuremath{\text{-grMod}}}
\newcommand{\Mod}{\ensuremath{\text{-Mod}}}
\newcommand{\Sym}{\ensuremath{\text{Sym}\,}}
\newcommand{\exterior}{\ensuremath{\Exter^\bullet}}


\newcommand{\coker}{\ensuremath{\text{coker}}}
\newcommand{\img}{\ensuremath{\text{im}}}
\newcommand{\Hom}{\ensuremath{\text{Hom}}}
\newcommand{\Ch}{\ensuremath{\mathscr{C}}}
\newcommand{\deri}{\ensuremath{\mathscr{D}}}
\newcommand{\kom}{\ensuremath{\mathscr{K}}}

\newcommand{\deritensor}{\ensuremath{\,\otimes^\mathbf{L}\,}}

\newcommand{\Proj}{\ensuremath{\text{Proj}}}
\newcommand{\Pn}{\ensuremath{{\mathbb{P}^n}}}
\newcommand{\coh}{\ensuremath{\text{coh-}}}

\newcommand{\gnab}{{\textexclamdown}}

% DOCUMENT
\title{(Co)Derived Equivalences in Algebra and Geometry}
\date{\vspace{-2.5em} Easter 2022}

\begin{document} 
\bibliographystyle{agsm}

\maketitle
\tableofcontents 

\vspace{1em}

Fix a field \(k\), and let \(X\) and \(\Xi\) be dual \(k\)-vector spaces of
dimension \(n+1\) with dual bases \((x_i)\) and \((\xi_i)\) respectively. The goal
of this exposition is to examine equivalences of various categories that arise
naturally in this setting from algebro-geometric constructions. In particular,
we look at chain complexes of 
\begin{enumerate}[(i)]
    \item modules over the symmetric algebra \(A:=
        \Sym^\bullet(X)\),
    \item modules over the exterior algebra \(A^!:= \exterior(\Xi)\),
    \item coherent sheaves over \(\Pn := \Proj(\Sym^\bullet (X))\), the
        projectivisation of \(\Xi\).
\end{enumerate}

The first hint at a correspondence between coherent sheaves on \(\Pn\) and
modules over the exterior algebra \(A^!=\exterior(\Xi)\) comes from the
following observation in \citeasnoun{bernstein_algebraic_1978}.

\begin{example}\label{bgg-thm1functor}
    Given a graded \(A^!\)-module \(N_\bullet\) and a non-zero vector \(\xi\in
    \Xi\), we have a complex of vector spaces 
    \[\mathscr{L}_\xi(N_\bullet): \qquad \cdots \rightarrow N_{1} \xrightarrow{\;\xi\;} N_0 \xrightarrow{\;\xi\;} N_{-1} \rightarrow \cdots. \]
    Writing \(\mathscr{L}^j\) for the vector bundle \(N_{-j}\otimes_k
    \mathscr{O}_{\Pn}(j)\), we can identify \(\mathscr{L}_\xi^j = N_{-j}\) with the
    fiber of \(\mathscr{L}^j\) at \([\xi]\in \Pn\) to obtain a complex 
    \[\mathscr{L}(N_\bullet): \qquad \cdots \rightarrow N_{1}\otimes_k
    \mathscr{O}_{\Pn}(-1) \longrightarrow N_{0}\otimes_k
    \mathscr{O}_{\Pn} \longrightarrow N_{-1}\otimes_k \mathscr{O}_{\Pn}(1)
    \rightarrow \cdots\] where the differential sends a section \(f\in
    N_{-j}\otimes_k \mathscr{O}(j)\) (which is a degree \(j\) homogeneous
    function with values in \(N_{-j}\)) to \(df = [\xi \mapsto \xi \cdot
    f(\xi)]\). We say the module \(N_\bullet\) is \textit{faithful} if the
    complex \(\mathcal{L}(N_\bullet)\) is exact everywhere except in degree
    \(0\), in which case we write \(\Phi(N_\bullet)\) for the vector bundle
    \(H^0(\mathcal{L}(N_\bullet))\). In this case, the natural map
    \(\mathcal{L}(N_\bullet) \rightarrow \Phi(N_\bullet)\) of omplexes
    (where we identify \(\Phi(N_\bullet)\) with the corresponding complex
    concentrated in degree \(0\)) gives a `resolution' of \(\Phi(N_\bullet)\) by
    powers of the Hopf bundle \(\mathscr{O}_{\Pn}(1)\).
\end{example}

\begin{example}\label{setstage}
In the simplest case when \(X, \Xi\) are one-dimensional, the data of a module
over \(A = k[x]\) involves a \(k\)-vector space \(M\) with a map
\(M\xrightarrow{x} M\) which can be seen as a cochain complex \(F(M)\) of
\(k\)-vector spaces with differential \(d\) of degree \(1\). The graded algebra
\(A^! = k[\xi]/(\xi^2)\) acts naturally on \(F(M)\) via the degree \(1\)
chain map \(F(M)\xrightarrow{\xi} F(M)\) given by
\[\begin{tikzcd}
    \cdots \arrow[r] & 0 \arrow[r] \arrow[rd] & M \arrow[r] \arrow[rd, "1"] & M
    \arrow[r] \arrow[rd] & \cdots &   \\ & \cdots \arrow[r] & M \arrow[r] & M
    \arrow[r] & 0 \arrow[r] & \cdots
\end{tikzcd}.\]
Consider the cochain complex of \(A\)-modules
\[\cdots \rightarrow 0 \rightarrow M\otimes_k A
\xrightarrow{ d\otimes 1 + \xi \otimes x} M\otimes_k A \rightarrow 0 \rightarrow
\cdots ,\]
concentrated in degrees \(-1\) and \(0\). This has the same underlying vector
spaces as the complex \(F(M)\otimes_k A\), but the differential has been
`twisted' to remember the \(A^!\)-action. This complex is exact everywhere
except in degree \(0\), where it has cohomology \(M\). Since the modules
appearing in it are free, we have recovered a free resolution of \(M\). 
\end{example}

This is the first example of what may be called \textit{Koszul duality}, a broad
term encompassing various equivalences across algebra, geometry, and representation
theory. The duality between symmetric and exterior algebras over finite
dimensional vector spaces was first studied by
\citeasnoun{bernstein_algebraic_1978}, who exhibit an adjunction between the
categories of cochain complexes of graded modules over \(A\) and \(A^!\).  
\begin{theorem*}
    There are adjoint functors
    \[ \Ch(A\grMod) \xleftrightarrows[F]{G} \Ch(A^!\grMod)\]
    such that any complex \(M^\bullet\) of graded
    \(A\)-modules has free resolution \(GF(M^\bullet)\), and any complex
    \(N^\bullet\) of graded \(A^!\)-modules has injective resolution
    \(FG(N^\bullet)\). 
\end{theorem*}
In \cref{sec-BGG}, we look at \possessivecite{eisenbud_sheaf_2003}
treatment of the Bernstein-Gel'fand-Gel'fand (BGG) correspondence described
above. To turn the adjunction into an equivalence of categories, we need to
employ the machinery of Verdier's \textit{derived categories}. Passing to the
corresponding derived categories of modules has the effect that all
\textit{quasi-isomorphisms} (i.e.\ chain maps that induce isomorphisms on
homology) become isomorphisms.

\citeasnoun{bernstein_algebraic_1978} use the correspondence between coherent
sheaves on \Pn\ and graded \(A\)-modules (see for example Chapter II of
\citeasnoun{hartshorne_algebraic_2008}) to describe the derived category
\(D^b(\coh\Pn)\) of projective \(n\)-space. In \ref{sec-cohPn} we use
\todo{finish this paragraph} 

\todo{insert paragraph about lefevre}

\section{Categories of complexes}\label{sec-trianglecat}

\subsection{Abelian categories}

\subsection{Bicomplexes}

\subsubsection{Spectral sequences}

\subsection{Homotopy and derived categories}

\subsubsection{Generators and exceptional sequences}

\subsubsection{Derived functors}

\section{Coherent sheaves on \Pn}

The projectivisation of \(\,\Xi\,\) is the \(k\)-scheme
\(\Pn=\Proj(\Sym^\bullet(X))\), where \(\Sym^\bullet(X)=k[x_0,...,x_n]\) is the
standard symmetric algebra on \(X\) graded by degree. We strengthen the
observation of \cref{bgg-thm1functor} to the following result, the celebrated
theorem of Beilinson. 
\begin{theorem*}[\citeasnoun{beilinson_coherent_1978}]
    The derived category \(\deri^b(\Pn)\) is generated by the exceptional
    sequence \[\langle \mathscr{O}(-n), \mathscr{O}(-n+1),..., \mathscr{O}(-1),
    \mathscr{O} \rangle.\]
\end{theorem*}
This section looks at \possessivecite{beilinson_coherent_1978} original
proof, following the treatment in \citeasnoun{caldararu_derived_2005} and
\citeasnoun{carbone_resolution_nodate}.  There are two key ideas involved-- the
first is that the identity functor on \(\deri(\coh\Pn)\) admits a factorisation
\[\begin{tikzcd}[column sep=huge]
    \deri^b( \Pn\times \Pn) \arrow[r, "-\deritensor \mathscr{O}_\Delta"] 
    & \deri^b(\Pn\times \Pn) \arrow[d, "\mathbf{R}\pi_{1\ast}"] \\
    \deri^b(\Pn) \arrow[u,"\pi_2^\ast"] \arrow[r,"\text{id}"] &
    \deri^b(\Pn)
\end{tikzcd}\]
where \(\pi_1,\pi_2:\Pn\times \Pn\rightarrow \Pn\) are the projection maps, and
\(\mathscr{O}_\Delta\in \coh(\Pn \times \Pn)\) is the structure sheaf of the
diagonal subscheme. This follows from the geometric theory of a
\textit{Fourier-Mukai transform} associated to a pair of schemes
\(\mathscr{X},\mathscr{Y}\), and we briefly sketch the construction in
\cref{fouriermukai}. 

The second observation, called \textit{Beilinson's resolution of the diagonal},
follows from the algebraic theory of \textit{Koszul resolutions} and shows that
that \(\mathscr{O}_\Delta\) admits a resolution by locally free sheaves of the
form \(\pi_1^\ast(\Omega^i(i))\otimes \pi_2^\ast (\mathscr{O}(-i))\), where
\(\Omega\) is the sheaf of differentials on \Pn. Combined with the factorisation
of identity, this provides an algorithm to resolve any coherent sheaf on \Pn in
terms of the \(\mathscr{O}(i)\) thus proving Beilinson's result.

\subsection{Fourier-Mukai transforms}\label{fouriermukai}

The material in this section is from \citeasnoun{huybrechts_fourier-mukai_2006},
who covers the topic in great detail. Given two smooth projective \(k\)-schemes
\(\mathbf{X_1}\) and \(\mathbf{X_2}\), we associate to each object
\(\mathscr{E}\in \deri^b(\mathbf{X_1}\times \mathbf{X_2})\) an exact functor
\(\Phi_\mathscr{E}: \deri^b(\mathbf{X_1})\rightarrow \deri^b(\mathbf{X_2})\) as
follows. 

\begin{defn} The \textit{Fourier-Mukai transform with kernel \(\mathscr{E}\)} of
    a complex \(\mathscr{A}\in \deri^b(\mathbf{X_1})\) is defined as 
    \[\Phi_\mathscr{E}(\mathscr{A})= \mathbf{R}\pi_{1\ast}(\pi_2^\ast
    \mathscr{A}\deritensor \mathscr{E}) \quad \in \deri^b(\mathbf{X_2}).\]
\end{defn}

Here \(\pi_i: \mathbf{X_1}\times \mathbf{X_2}\rightarrow \mathbf{X_i}\)
(\(i=1,2\)) be the projection maps, these are flat so the pullback functors
\(\pi_i^\ast\) are exact and need no derivation. Being the composition of three
exact functors, the Fourier-Mukai transform \(\Phi_\mathscr{E}\) is an exact
functor. Moreover, the dependence on the kernel is functorial-- for a fixed \(\mathscr{A}\in \deri^b(\mathbf{X_1})\), the map
\begin{align*} 
    \Phi_-(\mathscr{A}):\quad \deri^b(\mathbf{X_1}\times \mathbf{X_2})
    &\longrightarrow \deri^b(\mathbf{X_2})\\
    \mathscr{E}&\longmapsto \Phi_\mathscr{E}(\mathscr{A})
\end{align*}
is the composite \(\pi_{1\ast}(\pi_2^\ast\mathscr{A}\deritensor -)\), hence is
an exact functor. 

The name comes from the following analogy with functional analysis-- given a
finite-dimensional vector space \(X\) and its dual \(\Xi\), to any smooth
function \(E(x,\xi): X\times \Xi \rightarrow \mathbb{C}\) we can associate a
linear map \({\phi_E: L^2(X)\rightarrow L^2(Y)}\) between the spaces of
square-integrable functions, given by \({f\mapsto \int_X f(x)E(x,\xi)dx}\). If
\({E(x,\xi)= e^{2\pi i \langle x,\xi\rangle}}\), then \(\phi_E\) is an
isomorphism called the \textit{Fourier transform}. Similarly, the Fourier-Mukai
transform yields interesting functors based on choice of \(\mathscr{E}\).

\begin{example}
    If \(\mathbf{X}_1 = \mathbf{X}_2 = \mathbf{X}\) and
    \(\mathbf{X}\xhookrightarrow{\iota} \mathbf{X}\times \mathbf{X}\) is the
    diagonal inclusion, then we can consider the Fourier-Mukai transform with
    kernel \(\mathscr{O}_\Delta = \iota_\ast \mathscr{O}_X\), the structure
    sheaf of the diagonal subscheme. Since \(\iota\) is a closed immersion, the
    pushforward \(\iota_\ast\) is exact and \(\mathscr\mathbf{R}\iota_\ast =
    \iota_\ast\) as derived functors. Hence \(\mathscr{O}_\Delta =
    \mathbf{R}\iota_\ast \mathscr{O}_X\) in \(\deri^b(\mathbf{X})\), and we can
    use the projection formula to get
    \begin{align*}
        \Phi_{\mathscr{O}_\Delta}(\mathscr{A}) 
        &= \mathbf{R}\pi_{1\ast}(\pi_2^\ast \mathscr{A}\deritensor
        \mathbf{R}\iota_\ast \mathscr{O}_X) \\ 
        &= \mathbf{R}\pi_{1\ast} \circ \mathbf{R}\iota_\ast
        (\mathbf{L}\iota^\ast\,\pi_2^\ast \mathscr{A}\deritensor
        \mathscr{O}_X)  \\
        &= \mathbf{R}(\pi_1 \circ \iota)_\ast (\mathbf{L}(\pi_2\circ \iota)^\ast
        \mathscr{A} \deritensor \mathscr{O}_X)\\
        &= \mathscr{A}\deritensor \mathscr{O}_X.
    \end{align*}
    But \(\mathscr{O}_X\) is a locally free sheaf so the functor \((-\deritensor
    \mathscr{O}_X)\) is the same as \((-\otimes\mathscr{O}_X)\), which is
    identity.  In other words, the Fourier-Mukai transform with kernel
    \(\mathscr{O}_\Delta\) is the identity functor. 

    Replacing the trivial bundle \(\mathscr{O}_X\) in the above computation
    with some other line bundle \(\mathscr{L}\) on \(\mathbf{X}\), we see that
    the derived functor \((- \otimes \mathscr{L})\) is the Fourier-Mukai
    transform with kernel \(\iota_\ast \mathscr{L}\). Similarly, one can show
    that the Fourier-Mukai kernel \(\mathscr{O}_\Delta[1]\) yields the shift
    functor \(\mathscr{A}\mapsto \mathscr{A}[1]\). Thus Fourier-Mukai transforms
    generalise many familiar constructions. It is in fact a theorem of Orlov
    that any fully faithful exact functor \(\deri^b(\mathbf{X}_1)\rightarrow
    \deri^b(\mathbf{X}_2)\) that admits adjoints must arise as the Fourier-Mukai
    transform for some kernel determined uniquely up to isomorphism. 
\end{example}

\subsection{Koszul resolutions}

Given a ring \(A\) and a sequence \((a_0,...,a_n)\) of elements in \(A\), the
associated \textit{Koszul complex} is a very useful construction which detects
various homological properties of the ring, and often yields free resolutions of
the \(A\)-module \(A/(a_0,...,a_n)\). The construction and theory of Koszul
complexes is treated in its full generality in
\citeasnoun{eisenbud_commutative_1995}; here we only study the behaviour in two
special cases we use to resolve the diagonal as Beilinson did-- the first is
when \((a_0,...,a_n)\) generate the unit ideal, and the second is when they form
a regular sequence.  
\begin{defn}\label{koszulcomplexdefn}
Given a ring \(A\) and a sequence \((a_0,...,a_n)\) of elements in \(A\), the
associated \textit{Koszul complex} is the complex of \(A\)-modules given by
\begin{gather*}
    K_A(a_0,...,a_n) : \qquad 0\rightarrow \Exter_A^{n+1}(A^{n+1}) \rightarrow
    \Exter_A^n(A^{n+1}) \rightarrow
    \cdots \rightarrow \Exter_A^2(A^{n+1})\rightarrow A^{n+1} \rightarrow
    A \rightarrow 0  \\ 
    \qquad \qquad \qquad \qquad \qquad 
    d(e_{\alpha_1}\wedge...\wedge e_{\alpha_i}) = \sum_j
    (-1)^{i+j+1}a_{\alpha_j}\cdot (e_{\alpha_1} \wedge ...\hat{e}_{\alpha_j}...
    \wedge e_{\alpha_i})
\end{gather*}
where \(e_0,...,e_n\) are the standard generators of \(A^{n+1}\), and
\(\,\hat{\cdot}\,\) denotes omission of a term. We put the term
\(\Exter^i_A(A^{n+1})\) in differential degree \(-i\).
\end{defn}

Observe that the modules appearing in \(K_A(a_0,...,a_n)\) are free, so
acyclic Koszul complexes yield free \(A\)-resolutions. In the simplest case when
the sequence contains a single element \(a_0\), the Koszul complex is given by 
\[K(a_0):\qquad 0\rightarrow A \xrightarrow{\;a_0\; }A\rightarrow 0,\]
so it is exact if and only if \(a_0\) is a unit in \(A\). This result
generalises to sequences \((a_0,...,a_n)\) that generate the unit ideal.

\begin{prop}
    If \(A\) is a ring and \((a_0,...,a_n)=A\), then the
    Koszul complex \(K_A(a_0,...,a_n)\) is exact everywhere.
    \begin{proof}
        We show that the the identity on \(K_A(a_0,...,a_n)\) is chain
        homotopic to the zero morphism. By assumption, there are elements
        \(\lambda_0,...,\lambda_n\in A\) such that \(\sum_i \lambda_i a_i =-1\).
        Then consider the map given by
        \begin{align*}
            h: \Exter^i_A(A^{n+1})&\rightarrow \Exter^{i+1}_A(A^{n+1})\\
            h(e)&= \sum_j \lambda_j e\wedge e_j
        \end{align*}
        A straightforward basis-wise check shows \(d\circ h + h\circ d =
        \text{id}\), showing \(h\) is the required chain homotopy. Since
        homotopic chain-maps induce the same map on homology, we must have that
        \(K_A(a_0,...,a_n)\) is exact.
    \end{proof}
\end{prop}

Looking again at the Koszul complex for a single element \(a_0\in A\),
we have that \(H^1(K(a_0))=0\) if and only if \(a_0\) is not a
zero-divisor in \(A\)-- in this case the complex is a free resolution of
\(A/(a_0)\).  Recall that \((a_0,...,a_n)\) is an \(A\)-\textit{regular
sequence} if \(a_0\) is not a zero-divisor in \(A\), and for every \(0\leq i <
n\), \(a_{i+1}\) is not a zero-divisor for the module \(A/(a_0,...,a_i)\). Then
\citeasnoun{eisenbud_commutative_1995} proves that whenever \((a_0,...,a_n)\) is
a regular sequence, the associated Koszul complex is exact everywhere except in
degree \(0\) where it has cohomology \(A/(a_0,...,a_n)\). We prove a special
case of the result, when \(A\) is a polynomial algebra and \(a_0,...,a_n\) are
the indeterminates.

\begin{prop}[\citeasnoun{loday_algebraic_2012}]\label{prop-koszulcomplex-exactness}
    Suppose \(R\) contains a field of characteristic \(0\). Then
    for the polynomial ring \(A=R[x_0,...,x_n]\), we have 
    \[H^i(K_A(x_0,...,x_n)) = \begin{cases}
        R, &i=0\\
        0, &\text{otherwise}
    \end{cases}.\]

    \begin{proof} 
        Write \(M\) for the free \(R\)-module generated by \(x_0,...,x_n\). Then
        we can identify \(\Exter_A^\bullet(A^{n+1})\) with the algebra
        \(\Exter^\bullet_R(M)\otimes_k A\), graded so that \(M\) lies in degree
        \(1\). Considering \(A\) as an \(R\)-algebra graded by degree, we see
        that \(\mathbf{K}=K_A(x_0,...,x_n)\) is a complex of graded
        \(R\)-algebras given by 
        \begin{gather*} \label{k-koszul-strand}
            \mathbf{K}:\qquad 0\rightarrow \Exter^{n+1}_R(M)\otimes_R
            A\langle{-n-1}\rangle \rightarrow \Exter^n_R(M)\otimes_R
            A\langle{-n}\rangle \rightarrow \cdots \rightarrow A \rightarrow 0
            \\ 
            \qquad \qquad d ((x_{\alpha_1}\wedge...\wedge x_{\alpha_i})\otimes
            a) = \sum_j
            (-1)^{i+j+1}(x_{\alpha_1}\wedge...\hat{x}_{\alpha_j}...\wedge
            x_{\alpha_i}) \otimes ax_{\alpha_j}.
        \end{gather*} 
        Note the differential \(d\) preserves internal grading, so we can write the
        complex above as a direct sum \(\mathbf{K} = \bigoplus_r \mathbf{K}_r\)
        where \(\mathbf{K}_r\) is the complex of \(R\)-modules formed at
        Adam's degree \(r\) (called the \(r\)th \textit{strand} of
        \(\mathbf{K}\)). Since cohomology is an additive functor, we have
        \(H^i(\mathbf{K})=\bigoplus_r H^i(\mathbf{K}_r)\).

        Now the strands in negative degrees vanish everywhere, and the
        \(\mathbf{K}_0\) has the module \(R\) concentrated in differential degree
        \(0\).  Thus it suffices to prove every other strand is exact. We do
        this by showing that the identity map on \(\mathbf{K}_r\) is
        nullhomotopic whenever \(r > 0\). In this case, we know by assumption
        that \(r\in R\) is a unit so consider the map
        \begin{align*} 
            h: \Exter^i(M) \otimes_k A_{r-i} &\rightarrow
            \Exter^{i+1}(M) \otimes_k A_{r-i-1} \\
            h\left(m\otimes (x_{\beta_1}...x_{\beta_{r-i}})\right)&= -\frac{1}{r}
            \sum_{j}(m \wedge x_{\beta_j}) \otimes (x_{\beta_1}...
            \hat{x}_{\beta_j}...x_{\beta_{r-i}}).  
        \end{align*}
        A straightforward basis-wise check shows \(h\circ d + d\circ h =
        \text{id}\), i.e.\ \(h\) is a chain homotopy between \(\text{id}\)
        and the zero map.
    \end{proof}
\end{prop}

\paragraph{Koszul complexes in geometry.} We use Koszul complexes in the
following geometric setting-- on the affine scheme \(\mathbf{X}=\text{Spec }A\),
an \((n+1)\)-tuple \((a_0,...,a_n)\) in \(A\) can be seen as a global section of
the free sheaf \(\mathscr{E}=\mathscr{O}_\mathbf{X}^ {\oplus (n+1)}\). Then the
Koszul complex associated to \((a_0,...,a_n)\) yields a complex of coherent
sheaves, given by
\begin{gather*}
    \mathscr{K}_\mathbf{X}(s): \qquad 0\rightarrow \Exter^{n+1}\mathscr{E}^\vee
    \rightarrow \Exter^{n}\mathscr{E}^\vee \rightarrow\cdots \rightarrow
    \Exter^2\mathscr{E}^\vee \rightarrow \mathscr{O}_\mathbf{X}\rightarrow
    \mathscr{O}_{\mathbb{V}(s)}\rightarrow 0
\end{gather*}
where \(\mathbb{V}(s)\) is the zero locus of \(s\), i.e.\ the
closed subscheme corresponding to the ideal \((a_0,...,a_n)\). If
\((a_0,...,a_n)\) is a regular sequence then we have shown that the complex
\(\mathscr{K}_\mathbf{X}(s)\) is exact, giving a locally free resolution of
\(\mathscr{O}_{\mathbb{V}(s)}\).

The construction automatically extends to arbitrary schemes \(\mathbf{X}\)--
given global section of a locally free sheaf \(\mathscr{E}\in \coh\mathbf{X}\),
we can cover \(\mathbf{X}\) by affine open subschemes
\(\mathbf{X}=\bigcup_\alpha \mathbf{U}_\alpha\,\); then the associated Koszul
complexes \(\mathscr{K}_\mathbf{U_\alpha}(s|_\mathbf{U_\alpha})\) glue to give a
Koszul complex \(\mathscr{K}_\mathbf{X}(s)\) of coherent sheaves on
\(\mathbf{X}\). If \(s\) yields a regular sequence on each affine patch, then it
is immediate that the Koszul complex associated to \(s\) is a
locally free resolution of \(\mathscr{O}_{\mathbb{V}(s)}\).

\subsection{Beilinson's theorem}

We briefly discuss the sheaves involved before proving the
existence of Beilinson's resolution.

\paragraph{Sheaves on \Pn.} Writing \(x_{\alpha/0}=x_\alpha/x_0\) (\(1\leq \alpha
\leq n\)), we see that the standard affine patch \(\mathbf{U}_0 = \Pn\setminus
\mathbb{V}(x_0)\) has coordinate ring \(A^0=k[x_{1/0},...,x_{n/0}]\). Serre's
correspondence between graded \(A\)-modules and coherent sheaves on \Pn (see
Section II.5 of \citeasnoun{hartshorne_algebraic_2008}) asserts
that any \(\mathscr{M}\in \coh\Pn\) is the sheafification of some \(M\in
A\grMod\). Restricted to to the affine piece \(\mathbf{U}_0\), such a sheaf is
completely determined by the module of its global sections-- we will view this
\(A^0\)-module \(M^0\) as the degree \(0\) piece of the localisation
\(M[\frac{1}{x_0}]\). The transition functions for the sheaf come from the
natural isomorphisms \(M^0[\frac{x_0}{x_{\alpha}}]\cong
M^\alpha[\frac{x_\alpha}{x_0}]\).

As usual, \(\mathscr{O}_\Pn(j)\) denotes the \(j\)th twist of the structure
sheaf on \(\Pn\), which corresponds to the graded module \(A\langle j \rangle\).
On the patch \(\mathbf{U}_0\), the corresponding \(A^0\)-module \(A^0\langle
j\rangle\) is free of rank \(1\), with distinguished generator \(x_0^j\).

The cotangent sheaf \(\Omega\) of \Pn\, can be defined
via the \textit{Euler exact sequence} (see Section II.8 of
\citeasnoun{hartshorne_algebraic_2008}) as the kernel of 
\[\mathscr{O}_\Pn(-1)^{\oplus(n+1)}\longrightarrow \mathscr{O}_\Pn\,; \quad
(s_0,...,s_n)\longmapsto x_0s_0+...+x_ns_n.\]
Since the twisted sheaves \(\mathscr{O}_\Pn(j)\) are flat, we have for each \(j\in
\mathbb{Z}\) an exact sequence 
\[0\rightarrow \Omega(j)\rightarrow
\mathscr{O}_{\Pn}(j-1)^{\oplus(n+1)}\longrightarrow
\mathscr{O}_\Pn(j)\longrightarrow 0.\]
Writing \(dx_{\alpha/0}= (x_0e_\alpha - x_\alpha e_0)/x_0^2\) where
\(e_0,...,e_n\) are the standard generators of \(A^{n+1}\), we see that the
\(A^0\)-module corresponding to \(\Omega(j)|_{\mathbf{U}_0}\) is freely
generated by \(x^jdx_{1/0},...,x^jdx_{n/0}\). Write \(D^0\langle j\rangle\)
for this module. In particular, note that \(\Omega\) and its twists are
rank \(n\) locally free sheaves.

We write \(\Omega^i(j)\) for \(\Exter^i\Omega(j)\), the \(j\)th twist of the
sheaf of \(i\)-forms on \(\Pn\). On the affine patch \(\mathbf{U}_0\), this
corresponds to the module \(\Exter^i_{A^0}(D^0\langle j\rangle)\).

\paragraph{Resolution of the diagonal.} Let \(\iota: \Pn \rightarrow \Pn\times
\Pn\) be the inclusion of the diagonal subscheme \(\Delta\), and write \(\pi_1,
\pi_2:\Pn \times \Pn \rightarrow \Pn\) for the two coordinate projections.
We now prove that \(\mathscr{O}_\Delta =\iota_\ast \mathscr{O}_\Pn\) admits a
locally free resolution by sheaves of the form \(\pi_1^\ast\mathscr{O}(-j)\otimes
\pi_2^\ast\Omega^j(j)\). We build the resolution locally in the form of a Koszul
complex on each standard affine patch.

\begin{theorem}[\citeasnoun{beilinson_coherent_1978}]
    There is an exact sequence in \(\coh(\Pn \times \Pn)\) of the form 
    \[0\rightarrow \pi_1^\ast \mathscr{O}(-n)\otimes \pi_2^\ast \Omega^n(n)
    \rightarrow \cdots \rightarrow \pi_1^\ast \mathscr{O}(-1)\otimes \pi_2^\ast
    \Omega(1) \rightarrow \mathscr{O}_{\Pn\times \Pn} \rightarrow \mathscr{O}_\Delta
    \rightarrow 0.\]
    \begin{proof}
        Write \((x_0:...:x_n)\) and \((y_0:...:y_n)\) for the homogeneous
        coordinates on the two copies of \Pn, and let
        \(\mathbf{U}_\alpha=\Pn\setminus \mathbb{V}(x_\alpha)\) and
        \(\mathbf{V}_\alpha=\Pn\setminus \mathbb{V}(y_\alpha)\) be the standard
        affine patches. We denote modules corresponding to sheaves on
        \(\mathbf{U}_\alpha\) with a subscript \(x\), and on
        \(\mathbf{V}_\alpha\) with a subscript \(y\). Thus \(\mathbf{V}_\alpha\)
        has coordinate ring \(A^0_y \cong k[y_{1/0},...,y_{n/0}]\).

        We restrict to the affine open \(\mathbf{U}_0\times
        \mathbf{V}_0\subset \Pn\times \Pn\), which has coordinate ring
        \[A^{00}\;=\;A_x^0\otimes_k A_y^0 \;\cong\;
        k[x_{1/0},...,x_{n/0},y_{1/0},...,y_{n/0}].\] 

        Then the pullback
        \(\pi_1^\ast\mathscr{O}(-j)\) corresponds to the \(A^{00}\)-module
        \(A^0_x\langle -j\rangle\otimes_k A^0_y\), which is freely generated by
        \(x_0^{-j}\). Likewise, the pullback \(\pi_2^\ast \Omega^j(j)\) comes
        from the module \(\Exter^jD^0_y(j)\otimes_k A^0_x\), which is freely
        generated over \(A^{00}\) by elements of the form
        \(dy_{{\alpha_1}/n}\wedge...\wedge dy_{\alpha_i/n}\).
    \end{proof}
\end{theorem}

\section{The Bernstein-Gel'fand-Gel'fand correspondence}\label{sec-BGG}

\todo{describe section}

\subsection{Data}

\subsubsection{Symmetric and exterior algebras}

Given an \(n+1\)-dimensional \(k\)-vector space \(V\),
the \textit{tensor algebra} is the \(k\)-vector space 
\[ T(V) = k \oplus \bigoplus_{i\geq 1}(\underbrace{V\otimes_k V\otimes_k ...
\otimes_k V}_{i\text{ times}})\] 
with a product \(\nabla: T(V)\otimes T(V) \rightarrow T(V)\) induced by the
natural identifications \(V^{\otimes i}\otimes V^{\otimes j} \simrightarrow
V^{\otimes(i+j)}\). This is an associative algebra with a natural
\(\mathbb{Z}_{\geq 0}\)-grading. The \textit{symmetric algebra}
\(\Sym^\bullet(V)\) and the \textit{exterior algebra} \(\exterior(V)\) are then
the graded algebras defined as quotients of \(T(V)\) by certain two-sided
ideals, namely
\[\Sym^\bullet(V) = \frac{T(V)}{(x\otimes y - y\otimes x \;|\; x,y\in V)},
\qquad \exterior(V) = \frac{T(V)}{(x\otimes x \;|\; x\in V)}.\]
Since the ideals are generated by homogeneous elements, these algebras
inherit gradings from \(T(V)\).

We continue to use \(\nabla\) for the product morphism on either algebra,
though the corresponding bilinear map on \(\exterior{V}\) is often written
\(\wedge\).

\begin{remark} 
    We can repeat the above constructions in the category of \(R\)-modules for
    any ring \(R\). In this case, we write \(T_R(M),\; \Sym^\bullet_R(M),\;
    \Lambda^\bullet_R(M)\) respectively for the tensor, symmetric, and exterior
    algebras over \(M\in R\Mod\). In particular,
    \[ T(M) = R \oplus \bigoplus_{i\geq 1}(\underbrace{M\otimes_R M\otimes_R ...
    \otimes_R M}_{i\text{ times}}).\]
\end{remark}

Since we are primarily concerned with the algebras \(A=\Sym^\bullet(X)\) and
\(A^!=\exterior(\Xi)\), we redefine the grading on \(A^!\) as
\(A^!_{-i}=\Lambda^i \Xi\). This amounts to a change of sign from the usual
grading, but the convention ensures that the dual vector spaces \(X\) and
\(\Xi\) lie in degrees \(1\) and \(-1\) in their respective algebras.

\subsubsection{The exterior coalgebra}
\label{subsubsec-exteriorcoalgebra}

The \textit{exterior coalgebra} on \(\Xi\) is defined as the linear dual of
\(A^!\), written \(A^\gnab := \Hom_k(A^!, k)\). \(A^\gnab\) has the
\(\mathbb{Z}\)-grading \(A^\gnab_i=\Hom_k(A^!_{-i}, k)\) and is naturally an
\(A^!\)-module via \(a \cdot f(a') = (-1)^{\text{deg}\, a} f(a\wedge a')\) for
\(a\in A^!\) homogeneous, \(f\in \Hom(A^!,k)\). Moreover, for any \(k\)-vector
space \(N\) we have the natural isomorphism of \(A^!\)-modules \(\Hom_{k}(A^!,
N)\cong A^\gnab \otimes_k N\).  

Choosing a basis \(x_i\) for \(X\) fixes an isomorphism \(X\cong\Hom_k(\Xi,
k)=A^\gnab_1\), which can be extended to get the isomorphism of graded \(k\)-vector
spaces
\[A^\gnab = \bigoplus_i \Hom_k(\Lambda^i\Xi, k) \cong \bigoplus_i \Lambda^i X =
\exterior(X).\] 
Write \(\tau:A^\gnab \rightarrow A\) for the \(k\)-linear map which identifies
the subspaces of \(A^\gnab\) and \(A\) corresponding to \(X\), and is \(0\)
elsewhere. 


\paragraph{The coproduct on \(A^\gnab\).}
Being the linear dual of a finite dimensional algebra, \(A^\gnab\) has a natural
(coassociative counital) coalgebra structure which comes from dualising the
(associative unital) product \({\nabla: A^!\otimes_k A^!\rightarrow A^!}\). This
is called the \textit{shuffle coproduct}, and we give an explicit description of
it as follows. Given a collection of indices \(\underline{\alpha}=\{\alpha_1\,<\,
...\,<\,\alpha_i\}\subseteq \{0,...,n\}\), write \(x_{\underline{\alpha}}\) for the
standard basis element of \(A^\gnab\) given by \({x_{\alpha_1}\wedge
x_{\alpha_2} \wedge ... \wedge x_{\alpha_i}}\) (in particular,
\(x_\emptyset = 1\)). The vector \(\xi_{\underline{\alpha}}\) is defined
similarly. We say a tuple \((\underline{\beta}, \underline{\beta'})\) of subsets
is a \textit{break} of \(\underline{\alpha}\) if \((\beta_1\,<...<\,\beta_p,
\beta'_1\,<...<\,\beta'_q)\) is a permutation of \((\alpha_1\,<...<\,
\alpha_i)\) (in other words, \(\underline{\alpha} = \underline{\beta} \sqcup
\underline{\beta'}\)). The \textit{sign} of this break, written \(\langle
\underline{\beta}, \underline{\beta'}\rangle\), is defined to be the sign of the
corresponding permutation. Thus we have have 
\[\nabla(x_{\underline{\beta}} \otimes
    x_{\underline{\beta'}})=x_{\underline{\beta}} \wedge
    x_{\underline{\beta'}}=\langle \underline{\beta}, \underline{\beta'}\rangle
\, x_{\underline{\alpha}}.\] 
This allows us to write the coproduct on \(A^\gnab\) as 
\[\Delta(x_{\underline{\alpha}}) =
    \smashoperator[r]{\sum\limits_{(\underline{\beta},\underline{\beta'})\in
    \text{br}(\underline{\alpha})}} \langle \underline{\beta},
\underline{\beta'}\rangle \, x_{\underline{\beta}} \otimes
x_{\underline{\beta'}}\]
where \(\text{br}(\underline{\alpha})\) is the set of all breaks of
\(\underline{\alpha}\).  Recalling that \(A^\gnab\otimes_k A^\gnab\) is
\(\mathbb{Z}\)-graded with \(\bigoplus_{p+q=i}A^\gnab_p\otimes A^\gnab_q\) in
degree \(i\), we observe that the map \(\Delta\) respects grading hence
\(A^\gnab\) is a \textit{graded coalgebra}.

\subsubsection{Graded chain complexes}
\label{subsec-chaincomp}

Objects of \(\Ch(A\grMod)\) are chain complexes of graded \(A\)-modules in
which the differentials are morphisms in \(A\grMod\) (i.e.\ \(A\)-module
homomorphisms which preserve degree). Such an object can be viewed as a
\(\mathbb{Z}^2\)-graded \(k\)-vector space \(\bigoplus_{i,j}M^i_j\) with an
endomorphism \(d\) (the differential) such that 
\begin{enumerate}[]
    \item \(d\circ d=0\), 
    \item \(d\) has degree \((1,0)\) i.e.\ \(d(M^i_j)\subseteq M^{i+1}_j\),
        and
    \item for each \(i\in \mathbb{Z}\), \(M^i_\bullet = \bigoplus_{j}
        M^i_j\) is a graded \(A\)-module.  
\end{enumerate} 
Likewise, an object \(N\in \Ch(A^!\grMod)\) can be seen as a \(\mathbb{Z}^2\)-graded
\(k\)-vector space \(\oplus_{i,j}N^i_j\) with a differential \(\partial\) of
degree \((1,0)\). We shall use the two viewpoints on interchangeably, switching
between them whenever convenient to provide a clearer picture. In particular,
the ability to view a complex as a single module with additional structure
allows for cleaner definitions and proofs, see for instance \cref{adjunction}.

For a chain complex \(\mathbf{M}=\bigoplus_{i,j}M^i_j\), we say the lower
indices denote the \textit{internal} (or \textit{Adam's}) grading, while the
upper indices denote the \textit{differential} (or \textit{cohomological})
degree.  We use `\(\langle\cdot\rangle\)' to denote shifts in Adam's gradings,
continuing to use `\([\cdot]\)' to denote shifts in differential gradings.  Thus
for example we have \(M\langle q \rangle^i_j = M^i_{q+j}\).

\subsection{Twisted functors}

We now define additive functors 
\[\Ch(A\grMod) \xleftrightarrows[F]{G} \Ch(A^!\grMod)\]
on which the BGG correspondence is based. In the framework of
\(\mathbb{Z}^2\)-graded vector spaces described in \cref{subsec-chaincomp}, we have 
\[\bigoplus_{i,j}F(\mathbf{M})^i_j \cong \Hom_{k}\left(A^!,\,
\bigoplus_{p,q}M^p_q\right) = A^\gnab \otimes_k
\left(\bigoplus_{p,q}M^p_q\right), \qquad \bigoplus_{i,j}G(\mathbf{N})^i_j \cong
A\otimes_k \left(\bigoplus_{p,q}N^p_q\right).\] 
However, care is needed to define the gradings and differentials since, for
example, na\"ively applying the functor \(\Hom_k(A^!,-)\) would lose all
\(A\)-module structure. The key is to modify the na\"ive differential by adding a
`twist' as in \cref{setstage}.

\subsubsection{Defining the functor \(F\)} 
We first define \(F\) on the category \(A\grMod\), seen as the full subcategory of
\(\Ch(A\grMod)\) whose objects are complexes concentrated in differential degree
\(0\). If \(M^0_\bullet\) is a graded \(A\)-module, we define \(F(M^0_\bullet)\)
to be the chain complex of \(A^!\)-modules given by a
\begin{align*} 
    \cdots \rightarrow A^\gnab\langle -i \rangle \otimes_k M^0_i
    &\xrightarrow{\;\partial\;} A^\gnab \langle -i-1 \rangle \otimes_k
    M^0_{i+1} \rightarrow \cdots \\ 
    a\otimes m &\longmapsto \sum_\alpha \xi_\alpha a \otimes x_\alpha m.  
\end{align*} 
The module \(A^\gnab\langle -i \rangle \otimes_k M^0_i\) is naturally
isomorphic to \(\Hom_k(A^!\langle i \rangle, M^0_i)\) and inherits an Adam's
grading from \(A^\gnab\) with the vector space \(A^\gnab_{j-i}\otimes_k M^0_i\)
forming the \(j\)th graded piece. These shifts in grading have been chosen
precisely so that the differential \(\partial\) has degree \((1,0)\),  while
the graded commutativity of \(A^!\) implies \(\partial\circ \partial=0\).
Thus we indeed have a chain complex of \(A^!\)-modules. 

Given a morphism \(M^0_\bullet \rightarrow M^1_\bullet\) in \(A\grMod\),
the functoriality of tensor products induces \(A^!\)-module
homomorphisms \(A^\gnab\langle-i \rangle \otimes_k M^0_i \rightarrow
A^\gnab\langle-i\rangle \otimes_k M^0_i\) which are compatible with the
differentials (i.e.\ the natural squares commute). Thus we have an additive functor
\({F:A\grMod\rightarrow \Ch(A^!\grMod)}\).

To extend \(F\) to arbitrary chain complexes \(\mathbf{M}=(\bigoplus_{i,j}M^i_j,
d)\in \Ch(A\grMod)\), we observe that the functoriality of \(F\) gives us a (commuting) bicomplex
\begin{equation}\label{FM-totcomp}
    \begin{tikzcd}[row sep=large]
    \vdots 
           &            
           & 
           & \vdots 
           & \vdots 
           & \\
    F(M^{i+1}_\bullet) \arrow[u]
           & 
           & \cdots \arrow[r]
           & A^\gnab \langle -j \rangle\otimes_k M^{i+1}_j \arrow[r] \arrow[u]
           & A^\gnab \langle -j-1 \rangle\otimes_k M^{i+1}_{j+1} \arrow[r] \arrow[u]
           & \cdots \\
    F(M^{i}_\bullet) \arrow[u]
           & 
           & \cdots \arrow[r]
           & A^\gnab \langle -j \rangle\otimes_k M^{i}_j  \arrow[r] \arrow[u]
           & A^\gnab\langle -j-1 \rangle\otimes_k M^{i}_{j+1}  \arrow[r] \arrow[u]
           & \cdots \\
    \vdots \arrow[u]
           &            
           & 
           & \vdots \arrow[u] 
           & \vdots \arrow[u]
           &   
    \end{tikzcd}
\end{equation}
where the vertical maps are \(1\otimes d\). Define
\(F(\mathbf{M})\) to be the total complex of this bicomplex, i.e.\
\(F(\mathbf{M})\) is given by
\begin{gather}
    \cdots \rightarrow \bigoplus_{p+q=i}A^\gnab\langle -q \rangle \otimes_k
    M^p_q \xrightarrow{\;\partial\;} \bigoplus_{p+q=i+1}A^\gnab\langle -q
    \rangle\otimes_k M^p_q \rightarrow \cdots, \label{F-def} \\
    \partial: a\otimes m \longmapsto a\otimes dm + (-1)^{\#m}\sum_\alpha
    \xi_\alpha a \otimes x_\alpha m \nonumber
\end{gather}
where \(\#m\) is the differential degree of \(m\in \mathbf{M}\). It is clear
that each \(F(\textbf{M})^i_\bullet = \bigoplus_{p+q=i}A^\gnab\langle -q\rangle
\otimes_k M^p_q\) is a graded \(A^!\) module, and the signs introduced in the
total complex construction ensure \(\partial \circ \partial = 0\). An explicit
check confirms \(\partial\) has degree \((1,0)\), so we indeed have an object of
\(\Ch(A^!\grMod)\).

\paragraph{The twist using comodules.}
\label{comoduletwist}
Observe that the differential \(\partial\) differs from the na\"ive differential
\(1\otimes d\) on the tensor product by the horizontal maps, which are the
`twists' we have been alluding to. These have a nice description using the fact
that a graded module \(N_\bullet\in A^!\grMod\) has the structure of a
graded \(A^\gnab\)-comodule via the map
\begin{align*}
    \Delta: \quad N_\bullet &\longrightarrow N_\bullet \otimes_k A^\gnab \\
    n &\longmapsto \smashoperator[r]{\sum\limits_{\underline{\alpha}\subseteq
    \{0,...,n\}}} \xi_{\underline{\alpha}} n
    \; \otimes   x_{\underline{\alpha}}.
\end{align*}

Applying this idea to the \(A^!\)-modules \(A^\gnab\langle -i \rangle\), we get
a commuting square
\[\begin{tikzcd}[column sep=large, row sep=huge]  
    \bigoplus\limits_{u+v=j-q}A^\gnab_{u} \otimes_k A^\gnab_v \otimes_k
    M^{i-q}_q 
    \arrow[r, "1\otimes \tau \otimes 1"]
    & A^\gnab_{j-q-1} \otimes_k A_1 \otimes_k M^{i-q}_{q} 
    \arrow[d, "1\otimes \nabla"] \\
    A^\gnab_{j-q} \otimes_k M^{i-q}_q
    \arrow[u, "\Delta \otimes 1"] 
    \arrow[r, "(-1)^{i-q}(\partial \,- \,1\otimes d)"]
    &A^\gnab_{j-q-1} \otimes_k M^{i-q}_{q+1}
\end{tikzcd}\]

where \(\nabla: A\otimes_k M^{i-q}_\bullet \rightarrow M^{i-q}_\bullet\) defines the
\(A\)-module structure on \(M\), and \(\tau:A^\gnab \rightarrow A^!\) is the
morphism defined in \cref{subsubsec-exteriorcoalgebra} which identifies
\(A^\gnab_1\) with \(A_{1}\), annihilating other graded pieces. 

In summary, \(F(\mathbf{M})\) as a \(\mathbb{Z}^2\)-graded vector space is
simply \(A^\gnab \otimes_{k} \mathbf{M}\)  with \((i,j)\)th piece
\[F(\mathbf{M})^i_j = \bigoplus_{p+q=i}A^\gnab_{j-q}\otimes_k M^p_q\]
and differential given on \(A^\gnab_{j-q}\otimes_k M^p_q\) by 
\[1\otimes d + (-1)^{p}(1\otimes \nabla)\circ (1\otimes \tau \otimes
1)\circ(\Delta\otimes 1).\]



\subsubsection{The left adjoint to \(F\)}

The functor \(G:\Ch(A^!\grMod)\rightarrow \Ch(A\grMod)\) is analogously defined,
and maps the chain complex \(\mathbf{N}=(\bigoplus_{i,j}N, \partial)\) to \(G(\mathbf{N})\) given by
\begin{gather}
    \cdots \rightarrow \bigoplus_{p-q=i} N^p_q \otimes_k A\langle -q \rangle
    \xrightarrow{\;d\;} \bigoplus_{p-q=i+1}N^p_q \otimes_k A\langle -q \rangle
    \rightarrow \cdots  \label{def-G} \\
    d: n\otimes a \longmapsto \partial n\otimes a + (-1)^{\#n} \sum_\alpha
    \xi_\alpha n \otimes x_\alpha a \nonumber
\end{gather}
where \(\#n\) is the differential degree of \(n\in \mathbf{N}\). The Adam's
grading on each \(G(\mathbf{N})^i_\bullet\) is inherited from \(A\), and is
given by
\[G(\mathbf{N})^i_j =   \bigoplus_{p-q=i}N^p_q \; \otimes_k A_{j-q}.\]
Recalling that every \(A^!\)-module is a \(A^\gnab\)-comodule (see
\cref{comoduletwist}), we can use the comodule structure-map \(\Delta: N^i_\bullet
\rightarrow N^i_\bullet \otimes A^\gnab \) to define the differential on
\(N^p_q\otimes_k A_{j-q}\) as
\[\partial \otimes 1 + (-1)^{p}(1\otimes \nabla)\circ (1\otimes \tau \otimes
1)\circ(\Delta \otimes 1).\]

\paragraph{The adjunction.} 
Having defined the functors \(F\) and \(G\), we show that \(G\) is left adjoint
to \(F\). Spelled out this means given \(\mathbf{M}\in \Ch(A\grMod)\) and
\(\mathbf{N}\in \Ch(A^!\grMod)\), there is a natural isomorphism of abelian
groups
\[\Hom_{\Ch(A\grMod)}(G(\mathbf{N}), \mathbf{M}) \cong
\Hom_{\Ch(A^!\grMod)}(\mathbf{N}, F(\mathbf{M})).\]
At its heart this is a \(\otimes\)-\(\Hom\) adjunction, as we shall illustrate in
the special case of module categories below. 

\begin{lemma} \label{adjunction-simple}
    Given modules \(M\in A\Mod\) and \(N\in A^!\Mod\), there are natural
    isomorphisms of abelian groups
    \[\Hom_{A}(A\otimes_k N, M) \cong \Hom_k(N,M) \cong \Hom_{A^!}(N,
    \Hom_{k}(A^!,M)).\]
    \begin{proof}
        Consider the \((A,A^!)\)-bimodule \(T=A\otimes_k A^!\). Then the
        standard \(\otimes\)-\(\Hom\) adjunction for bimodules
        \cite{bourbaki_algebra_1989} gives us a natural isomorphism 
        \[\Hom_A(T \otimes_{A^!} N, M) \cong \Hom_{A^!}(N, \Hom_{A}(T,M)).\]
        Then observe that there are natural isomorphisms
        \[T \otimes_{A^!} N \cong A\otimes_k A^! \otimes_{A^!} N \cong A\otimes_k
        N, \qquad \Hom_A(T,M)\cong \Hom_A(A\otimes_k A^!, M)\cong \Hom_k(A^!,M).\]
        The isomorphism with \(\Hom_k(N,M)\) comes similarly from treating \(A\)
        as an \((A,k)\)-bimodule.
    \end{proof}
\end{lemma}

We now exhibit the general adjunction for \(F\) and \(G\), and it is here that
the flexibility of interpreting a chain complex \(\mathbf{M}\) of graded modules
as a single \(\mathbb{Z}^2\)-graded module \(\bigoplus_{i,j}M^i_j\) (see
\cref{subsec-chaincomp}) really comes handy.  Interpreting \(\Ch(A\grMod)\) as a
subcategory of \(A\Mod\) (likewise for \(A^!\)), we use \cref{adjunction-simple}
to identify \(\Hom_{\Ch(A\grMod)}(G(\mathbf{N}),\mathbf{M})\subset
\Hom_A(\mathbf{N}\otimes_k A, \mathbf{M})\) and
\(\Hom_{\Ch(A^!\grMod)}(\mathbf{N},F(\mathbf{M}))\subset \Hom_{A^!}(\mathbf{N},
\Hom_k(A^!,\mathbf{M}))\) with the same subgroup of
\(\Hom_k(\mathbf{N},\mathbf{M})\).

\begin{theorem}[\citeasnoun{bernstein_algebraic_1978}] \label{adjunction}
    The functor \(G\), from the category of complexes of graded \(A^!\)-modules
    to the category of complexes of graded \(A\)-modules, is a left adjoint to
    the functor \(F\).
    \begin{proof}
        Given \(\bar{\varphi}\in \Hom_{A}(G(\mathbf{N}),
        \mathbf{M})\), the corresponding map \(\varphi\in
        \Hom_k(\mathbf{N},\mathbf{M})\) found in \cref{adjunction-simple} is
        given by \(\varphi(n)=\bar\varphi(n\otimes 1)\). Thus \(\bar\varphi\)
        has degree \((0,0)\) if and only if \(\bar\varphi(N^i_j \otimes_k A_0)
        \subseteq M^{i-j}_j\), if and only if \(\varphi(N^i_j)\subseteq
        M^{i-j}_j\).  Moreover for \(n\in N^i_j\), direct computation shows
        \begin{align*}
            (d_M \circ \bar\varphi - \bar\varphi \circ d_{G(\mathbf{N})}) (n
            \otimes 1) \; = \; (d_M \circ \varphi - \varphi \circ
            \partial_N) (n) - (-1)^i \sum_\alpha x_\alpha \varphi(\xi_\alpha n),
        \end{align*}
        thus \(\bar\varphi\) is a morphism in \(\Ch(A\grMod)\) if and only if
        \begin{equation}\label{adjunctcondition}
            \varphi(N^i_j)\subseteq M^{i-j}_j, \quad \text{and} \quad d_M\circ
            \varphi - \varphi \circ \partial_N = \sum_\alpha x_\alpha \varphi
            \xi_\alpha        
        \end{equation}
        where we write \(\sum_\alpha x_\alpha\varphi \xi_\alpha\) for the map
        that takes \(n\in N^i_j\) to \((-1)^i\sum_\alpha x_\alpha
        \varphi(\xi_\alpha n)\).

        Likewise given \(\varphi^!\in \Hom_{A^!}(\mathbf{N},F(\mathbf{M}))\),
        repeating the above argument shows \(\varphi^!\) is an element of
        \(\Hom_{\Ch(A^!\grMod)}(\mathbf{N},F(\mathbf{M})\) if and only if
        the corresponding map \(\varphi \in \Hom_k(\mathbf{N},\mathbf{M})\)
        satisfies \eqref{adjunctcondition}. This shows that the isomorphisms
        given in \cref{adjunction-simple} restrict to isomorphisms
        \[\Hom_{\Ch(A\grMod)}(G(\mathbf{N}),\mathbf{M})\cong \{\varphi\in \Hom_k(\mathbf{N},\mathbf{M}) \text{ satisfying \eqref{adjunctcondition}}\} \cong \Hom_{\Ch(A^!\grMod)}(\mathbf{N},F(\mathbf{M})) \]
        thereby showing \(G\) is left adjoint to \(F\).
    \end{proof}
\end{theorem}

\subsection{Koszul resolutions}

Given a complex \(\mathbf{M}\in \Ch(A\grMod)\), the adjunction \(F\vdash G\)
takes the identity morphism
\[1_{F(\mathbf{M})}\in \Hom_{\Ch(A^!\grMod)}(F(\mathbf{M}), F(\mathbf{M}))\]
to a map 
\[\varepsilon_{\mathbf{M}}\in \Hom_{\Ch(A\grMod)}(G(F(\mathbf{M})), \mathbf{M}).\]
The natural transformation \(\varepsilon:G\circ F\rightarrow \mathbf{1}\) thus
obtained is called the \textit{counit} of the adjunction, and we say the
morphism \(\varepsilon_{\mathbf{M}}\) is the \textit{component} of the
transformation at \(\mathbf{M}\). Likewise, there is the dual notion called the
\textit{unit} of the adjunction, which is a natural transformation \(\eta:
\mathbf{1}\rightarrow F\circ G\) giving, for any \(\mathbf{N}\in
\Ch(A^!\grMod)\), a morphism \(\eta_\mathbf{N}:\mathbf{N}\rightarrow
F(G(\mathbf{N})).\) 

Begin with the following observation.

\begin{prop} \label{prop-F-2}
    The functor \(F\) maps elements of \(\Ch(A\grMod)\) to complexes of
    injective \(A^!\)-modules, and the functor \(G\) maps elements of
    \(\Ch(A^!\grMod)\) to complexes of free \(A\)-modules.
    \begin{proof}
        The statement for \(G\) is immediate from definition, so we prove that
        for any \(\mathbf{M}\in \Ch(A\grMod)\), the modules
        \(F(\mathbf{M})^i_\bullet\) are injective over \(A^!\).

        Recall from \citeasnoun{weibel_introduction_2003} (Proposition 2.3.10)
        that if \(R: \mathcal{B}\rightarrow\mathcal{A}\) is an additive functor
        which is right adjoint to an exact functor
        \(L:\mathcal{A}\rightarrow\mathcal{B}\), then for any injective object
        \(I\in \mathcal{B}\) the object \(R(I)\in \mathcal{A}\) is injective.
        Applying this to the pair of adjoint functors 
        \[R= \Hom_k(A^!, -): k\grMod \rightarrow A^!\grMod , \quad L =
        (-\otimes_k A^!): A^!\grMod \rightarrow k\grMod\] 
        and observing that \(L\) is exact since all \(k\)-vector spaces are flat
        over \(k\), see that \(R\) preserves injectives. But every \(k\)-vector
        space is also injective, so the \(A^!\)-modules \(A^\gnab\langle -q
        \rangle \otimes_k M^p_q \cong R(M^p_q)\) appearing in \eqref{F-def} are
        all injective. To conclude, observe that the \(k\)-algebra \(A^!\) is finite
        dimensional hence noetherian. By the theorem of Bass \& Papp (see
        \citeasnoun{lam_lectures_1999}, Theorem 3.46) which asserts that a ring
        is (left) noetherian if and only if any direct sum of injective
        modules over it is injective, we are done.
    \end{proof}
\end{prop}

We show that the component \(\varepsilon_\mathbf{M}:G(F(\mathbf{M}))\rightarrow
\mathbf{M}\) is, in fact, a free resolution of the complex \(\mathbf{M}\) and
dually, the component \(\eta_\mathbf{N}\) is an injective resolution of
\(\mathbf{N}\). A special but important case of this phenomenon is when the
complex is \[\cdots \rightarrow 0\longrightarrow k\longrightarrow 0\rightarrow
\cdots,\] and this will be central in proving the result for general complexes.

\subsubsection{The Koszul complex} 
The \(1\)-dimensional vector space \(k\) can be considered a graded \(A\)-module
concentrated degree \(0\), such that all \(x_i\in A\) annihilate \(k\). Then
\(F(k)\) is the complex \(0\rightarrow \Lambda^\bullet(X) \rightarrow 0\)
concentrated in differential degree \(0\). We compute the complex \(G(F(k))\) to
be 
\begin{gather} 
    0\rightarrow A^\gnab_{n+1}\otimes_k A\langle-n-1\rangle \rightarrow
    A^\gnab_n \otimes_k A\langle -n \rangle \rightarrow ... \rightarrow
    A^\gnab_1\otimes_k A\langle-1\rangle \rightarrow A^\gnab_0\otimes_k  A
    \rightarrow 0 \label{k-koszulcomplex} \\ 
    (x_{\alpha_1}\wedge...\wedge x_{\alpha_i})\otimes 1 \longmapsto
    \sum_{j}(-1)^{i+j-1} (x_{\alpha_1}\wedge ... \hat{x}_{\alpha_j} ... \wedge
    x_{\alpha_i}) \otimes x_{\alpha_j} \nonumber 
\end{gather} 
where \(\;\hat{\cdot}\;\) denotes omission of a term. 

Observing the term in differential degree \(i\) is isomorphic to
\(\Lambda^i_A (A^{n+1})\), we can recognise the above complex as the
\textit{Koszul complex} associated to the regular sequence
\((x_0,...,x_n)\in A\). Then standard results on Koszul complexes found in
\citeasnoun{eisenbud_commutative_1995} (Chapter 17 and relevant sections of
Appendix 2) show that this complex has cohomology \(0\) everywhere except in
degree \(0\), where the cohomology is \(A/(x_1,...,x_n)\cong k\). We provide
below a direct proof of the result in the special case when
\(\text{char}(k)=0\).


A similar argument shows that the complex \(F(G(k))\) is exact everywhere except
in degree \(0\), where it has cohomology \(k\). Thus we have resolutions  
\[G(F(k))\rightarrow k\rightarrow 0, \qquad 0\rightarrow k\rightarrow F(G(k))\] 
of \(k\) by free \(A\)-modules and by injective \(A^!\)-modules, respectively.
It is not hard to see that these maps are precisely the ones given by the counit
and the unit of adjunction. 

\subsubsection{Resolutions in general} 
We show that the counit (resp.\ unit) gives a free (resp.\ injective)
resolution, by first showing this is the case for graded modules (i.e.\
complexes concentrated in a single degree). When \(\mathbf{M}\) is a graded
\(A\)-module, we will show that the complex \(G(F(\mathbf{M}))\) is `built up'
from the tensor product of \(\mathbf{M}\) and the Koszul complex of \(k\). 

\begin{lemma}[\citeasnoun{eisenbud_sheaf_2003}]\label{cor-module-freeres}
    If \(\mathbf{M}\in \Ch(A\grMod)\) is a chain complex concentrated in
    differential degree \(0\), then the natural map
    \[\varepsilon_\mathbf{M}: G(F(\mathbf{M}))\rightarrow \mathbf{M}\] 
    is an epimorphism and induces an isomorphism on cohomology. Likewise, if
    \(\mathbf{N}\in \Ch(A^!\grMod)\) is a chain complex concentrated in
    differential degree \(0\), then the natural map 
    \[\eta_\mathbf{N}: \mathbf{N}\rightarrow F(G(\mathbf{N}))\]
    is a monomorphism and induces an isomorphism on cohomology.
    \begin{proof}
        We first show that the complex \(G(F(\mathbf{M}))\) has the same
        cohomology as the complex \(\mathbf{M}\). Direct computation shows
        \(G(F(\mathbf{M}))\) is given by 
        \begin{gather*} 
            \cdots \rightarrow \bigoplus_{p}
            A^\gnab_{-i}\otimes_k M^0_p \otimes_k A\langle i-p
            \rangle \longrightarrow \bigoplus_{p}
            A^\gnab_{-i-1}\otimes_k M^0_p \otimes_k A\langle i+1-p
            \rangle \rightarrow \cdots, \\ 
            a\otimes m \otimes b \longmapsto \sum_\alpha\xi_\alpha a \otimes
            x_\alpha m \otimes b \;+\; (-1)^{\text{deg } m} \sum_\alpha\xi_\alpha a
            \otimes m \otimes x_\alpha b 
        \end{gather*}
        so the strand of this in Adam's degree \(r\) is seen to be the total
        complex of the (commuting) bicomplex
        \begin{equation}\label{bicomplex-strand}
            \begin{tikzcd}
                   & \vdots \arrow[d]
                   & \vdots \arrow[d]
                   & \\
            \cdots \arrow[r]
                   & A^\gnab_{-i}\otimes_k M^0_p \otimes_k A_{i-p+r} 
                   \arrow[r] \arrow[d]
                   & A^\gnab_{-i-1}\otimes_k M^0_p \otimes_k A_{i-p+r+1} 
                   \arrow[r] \arrow[d]
                   & \cdots \\
            \cdots \arrow[r]
                   & A^\gnab_{-i-1}\otimes_k M^0_{p+1} \otimes_k A_{i-p+r}
                   \arrow[r] \arrow[d]
                   & A^\gnab_{-i-2}\otimes_k M^0_{p+1} \otimes_k A_{i-p+r+1}
                   \arrow[r] \arrow[d]
                   & \cdots \\
                   & \vdots
                   & \vdots 
                   & 
            \end{tikzcd}.\end{equation}
        Here \(p\)th row is obtained by applying \((-\otimes_k M^0_p)\) to the
        complex \(G(F(k))_{r-p}\), so is exact from
        \cref{prop-koszulcomplex-exactness} unless \(p=r\). Moreover, the
        \(r\,\)th row is 
        \[ \cdots 0 \rightarrow M^0_r \rightarrow 0 \rightarrow \cdots. \]  
        Thus first page of the spectral sequence (starting with horizontal
        cohomology) of \eqref{bicomplex-strand} is 
        \[\begin{tikzcd} 
                   & \vdots \arrow[d]
                   & \vdots \arrow[d]
                   & \\
            \cdots 
                   & M^0_r \arrow[d]
                   & 0 \arrow[d]
                   & \cdots \\
            \cdots  
                   & 0 \arrow[d]
                   & 0 \arrow[d]
                   & \cdots \\
                   & \vdots
                   & \vdots 
                   & 
        \end{tikzcd}.\]
        By the theorem on spectral sequences\todo{insert theorem}, we
        conclude that the total complex \(G(F(\mathbf{M}))_r\) has cohomology 
        \[H^k(G(F(\mathbf{M}))_r)=\begin{cases} M^0_r, &k=0 \\ 
        0, &\text{otherwise}\end{cases}.\]

        Now it suffices to show that the map
        \(\varepsilon_{\mathbf{M}}:G(F(\mathbf{M}))^0_r \rightarrow M^0_r\) is
        the cokernel of \({G(F(\mathbf{M}))^{-1}_r \rightarrow
        G(F(\mathbf{M}))^0_r}\). But this is immediate because the
        sequence 
        \[\begin{tikzcd}[row sep=tiny] 
            \bigoplus_p X\otimes_k M^0_p \otimes_k \Sym^{r-p-1}(X) 
            \arrow[r]
            & \bigoplus_p  M^0_p \otimes_k \Sym^{r-p}(X) 
            \arrow[r]
            & M^0_r 
            \arrow[r]
            & 0 \\
            x_\alpha \otimes m \otimes a
            \arrow[r, mapsto]
            & m \otimes x_\alpha a + (-1)^{\text{deg }m} x_\alpha m \otimes a
            & \\ 
            & m \otimes a  
            \arrow[r, mapsto]
            & (-1)^{\text{deg} m}am
        \end{tikzcd}\]
        is (split) exact.

        The analogous statement about graded \(A^!\)-modules follows from a
        similar calculation.
        \end{proof}
\end{lemma}

The argument to extend this result to all chain complexes is purely formal.

\begin{theorem}[\citeasnoun{eisenbud_sheaf_2003}]
    \label{thm-eisenbud-res}
    For any complex \(\mathbf{M}\in \Ch(A\grMod)\), the complex
    \(G(F(\mathbf{M}))\) is a free resolution of \(\mathbf{M}\) which surjects
    onto \(\mathbf{M}\), and for any complex \(\mathbf{N}\in \Ch(A^!\grMod)\),
    the complex \(F(G(\mathbf{N}))\) is an injective resolution of
    \(\mathbf{N}\) which \(\mathbf{N}\) injects into.
    \begin{proof}
        Given \(\mathbf{M}\in \Ch(A\grMod)\), the surjectivity of
        \(\varepsilon_\mathbf{M}: G(F(\mathbf{M}))\rightarrow \mathbf{M}\) can
        be checked on the level of underlying \(\mathbb{Z}^2\)-graded
        modules. The map is given on the \((i,j)\)th component by 
        \begin{align*}
            \bigoplus_{p,q} \Hom_k(A^!_{q-i}, M^q_{p-q}) \otimes_k A_{j-p+i}
            &\longrightarrow M^i_j \\
            f\otimes a \mapsto af(1)
        \end{align*} 
        \todo{check sign} 
        hence any \(m\in M^i_j\) can be written
        \(\varepsilon_\mathbf{M}(f_m\otimes 1)\) where \(f_m: A^!_{0}\rightarrow
        M^i_j\) is the function \(f_m(1)=m\). 

        To prove that the induced map on cohomology is an isomorphism we first
        reduce the problem to bounded complexes using formal properties of the
        functors, and then induct on the length of the complex to reduce our
        problem to \cref{cor-module-freeres}. The key properties we use are
        naturality of \(\varepsilon\), and the fact that \(G\), \(F\) and
        cohomology functors all preserve direct limits.\todo{prove}

        Any complex \(\mathbf{M}\in \Ch(A\grMod)\) can be written as the direct
        limit of bounded complexes \((\mathbf{M}^b)_{b\in B}\), giving us
        commuting diagrams
        \begin{equation} \label{diagram-boundednaturality}
            \begin{tikzcd}[row sep=large]
                G(F(\mathbf{M}^b)) \arrow[r]\arrow[d,
                "\varepsilon_{\mathbf{M}^b}"] & G(F(\mathbf{M})) \arrow[d,
                "\varepsilon_{\mathbf{M}}"] \arrow[r, Rightarrow, no head]  &
                \overset{\lim}{\rightarrow} G(F(\mathbf{M}^b)) \\ 
                \mathbf{M}^b \arrow[r] & \mathbf{M} \arrow[r, Rightarrow, no
                head] & \overset{\lim}{\rightarrow} \mathbf{M}^b
            \end{tikzcd}.
        \end{equation}
        Then applying the \(i\)th cohomology functor \(H^i\)
        to \eqref{diagram-boundednaturality} then shows that the map
        \(H^i(\varepsilon_\mathbf{M})\) is the limit of the maps
        \(\varepsilon_{\mathbf{M}^b}\), so to show
        \(H^i(\varepsilon_{\mathbf{M}})\) is an isomorphism it suffices to show
        all \(H^i(\varepsilon_{\mathbf{M}^b})\) are. Thus without loss o
        generality the complex \(\mathbf{M}\) is bounded. Since \(F\) and \(G\)
        respect translation in differential degree, say \(\mathbf{M}\) has form
        \begin{equation}\label{ses-induction}
            0\rightarrow M^0_\bullet \rightarrow ... \rightarrow M^d_\bullet
            \rightarrow 0.
        \end{equation}
        Let \(\mathbf{M}^d\) be the chain complex with \(M^d_\bullet\) in degree
        \(d\), and \(0\) elsewhere. We have a short exact sequence 
        \[0\longrightarrow \ker(\varphi) \longrightarrow \mathbf{M}
        \xrightarrow{\;\varphi\;} \mathbf{M}^d \longrightarrow 0\]
        where \(\varphi\) is the obvious map. The complex \(\ker(\varphi)\) is
        concentrated in degrees \(0\),..., \(d-1\). Applying the exact functor
        \(H^i\) to the diagram formed by the naturality squares of
        \(\varepsilon\) on \eqref{ses-induction} gives us a commutative diagram 
        \[\begin{tikzcd}[row sep=large]
            0 \arrow[r] 
            & H^i(G(F(\ker\varphi))) \arrow[r] \arrow[d,
            "H^i(\varepsilon_{\ker\varphi})"] 
            & H^i(G(F(\mathbf{M}))) \arrow[r] \arrow[d,
            "H^i(\varepsilon_\mathbf{M})"]
            & H^i(G(F(\mathbf{M}^d))) \arrow[r] \arrow[d, "(H^i
            \varepsilon_{\mathbf{M}^d})"]
            & 0 \\ 
            0 \arrow[r]
            & H^i(\ker(\varphi)) \arrow[r]
            & H^i(\mathbf{M}) \arrow[r]
            & H^i(\mathbf{M}^d) \arrow[r]
            & 0
        \end{tikzcd}\]
        where the rows are exact. By \cref{cor-module-freeres},
        \(H^i(\varepsilon_{\mathbf{M}^d})\) is an isomorphism. By the Five
        lemma, \(H^i(\varepsilon_\mathbf{M})\) is an isomorphism if and only if
        \(H^i(\varepsilon_{\ker(\varphi)})\) is. Since \(\ker(\varphi)\) is a
        strictly shorter complex than \(\mathbf{M}\), we are done.

        The analogous statement for \(F\circ G\) follows from a similar
        calculation.
    \end{proof}
\end{theorem}

Thus we have a formulaic (albeit inefficient-- the free \(A\)-module \(A\) is
resolved to an \(n\)-term free complex) method to compute resolutions of
complexes. 

\paragraph{Syzygies and regularity of modules.} We use the resolutions produced
in \cref{thm-eisenbud-res} to prove a classical result of commutative algebra--
Hilbert's syzygy theorem, and provide a way to compute the Castelnuovo-Mumford
regularity of modules. We briefly discuss the notions involved, and refer to
\citeasnoun{eisenbud_commutative_1995} for details.

Writing a graded \(A\)-module \(M\) in terms of generators and relations
produces a short exact sequence 
\[0\rightarrow S\rightarrow F \rightarrow M \rightarrow 0, \]
where \(F\) is a free module. The module \(S\) is unique up to direct sum with
a free module (i.e.\ if \(0\rightarrow S' \rightarrow F'
\rightarrow M \rightarrow 0\) is another such resolution then
there are free modules \(L\) and \(L'\) such that \(L\oplus S \cong L'\oplus
S'\)), and is called the \textit{first syzygy} of \(M\). Continuing
the process, we can write \(S\) in terms of generators and
relations and define the second syzygy of \(M\) to be the first
syzygy of \(S\). Thus the \(j\)th syzygy of \(M\) is the module \(S_j\) (up to direct sum with a free module) such that there is an exact sequence 
\[0\rightarrow S_j \rightarrow F_{j-1} \rightarrow ...
\rightarrow F_0 \rightarrow M \rightarrow 0\] 
where \(F_0,...,F_{j-1}\) are free modules. Note that if the \(j\)th syzygy of
\(M\) is free then \(M\) has a free resolution of length \(j+1\)-- thus syzygies
form a measure of the `complexity' of \(M\). This is made precise using the
notion of \textit{projective dimension}, defined as
\[\text{pd}(M)=\min\{j\;|\; \text{the }j\text{th syzygy module of \(M\) is free
or projective}\}.\]
Hilbert showed that the projective dimension of \(A\)-modules is bounded. The
resolution produced using \cref{thm-eisenbud-res} provides a immediate
constructive proof of this result.

\begin{cor}[Hilbert Syzygy Theorem]
    If \(M\) is a graded module over \(k[x_0,...,x_n]\), then the \(n+1\)st
    syzygy module of \(M\) is free. 
\end{cor}

In fact, this bound is strict-- for instance, the \(A\)-module \(k\) has
projective dimension \(n+1\). To see this, observe
that \eqref{k-koszulcomplex} allows us to compute \(\text{Ext}^{n+1}_A(k,k)\cong
k\) but by Lemma 4.1.6 of \citeasnoun{weibel_introduction_2003}, an \(A\)-module
\(M\) has a projective resolution of length \(\leq d\) if and only if
\(\text{Ext}^{d}_A(M,N)=0\) for all \(A\)-modules \(N\). Defining the
\textit{graded global dimension} of a graded ring \(R\) to be
\[\text{gr.gl.dim}(R)=\sup\,\{\text{pd}(M)\;|\;M\in R\grMod \},\]
we have thus shown that \(\text{gr.gl.dim}(k[x_0,...,x_n])=n+1\).

The notion of \textit{Castelnuovo-Mumford regularity} builds upon this,
putting a bound on the degrees of generators and relations of a finitely
generated graded \(A\)-module \(M\). We say \(M\) is \(m\)-regular if the
\(j\)th syzygy of \(m\) is generated in degrees \(\leq m+j\).
We state a homological characterisation of regularity, referring to
\citeasnoun{eisenbud_linear_1984} for a proof. 

\begin{theorem}[\citeasnoun{eisenbud_linear_1984}] For a finitely generated
    graded \(A\)-module \(M\), the following conditions are equivalent. 
    \begin{enumerate}
        \item \(M\) is \(m\)-regular.
        \item \(M_{\geq m} = \bigoplus_{i\geq m} M_i\) is generated by \(M_m\) and
            has a \textit{linear free resolution} (a free resolution in which
            the differentials are represented by matrices whose entries have
            degree \(\leq 1\).)
        \item \(M_{\geq m}\) is generated by \(M_m\) and
            \(\text{Tor}_A(M,k)^{j-i}_{j}=0\) for all \(j\) and all \(i>m\).
    \end{enumerate}
\end{theorem}

Using the Koszul complex \eqref{k-koszulcomplex}, we extend the result above to
the following. 

\begin{cor}[\citeasnoun{eisenbud_sheaf_2003}] \label{regularity-criterion} A
    finitely generated graded \(A\)-module \(M\) is \(m\)-regular if and only if
    \(M_{\geq m}\) is generated by \(M_m\) and the complex \(F(M)\) is exact at
    degrees \(>m\).
    \begin{proof}
        It suffices to show that the complex \(F(M)\) has cohomology \(H^i(F(M))_j=
        \text{Tor}_A(M,k)^{j-i}_{j}\). To see this, note that the Koszul complex
        \(G(F(k))\) given by \eqref{k-koszulcomplex} is a free resolution of
        \(k\). Then the complex \(M\otimes_A G(F(k))\) is given in differential
        degree \(i-j\) by 
        \[M\otimes_AA^\gnab_{j-i}\otimes_k A\langle i-j \rangle \cong
        A^\gnab_{j-i} \otimes_k M\langle i-j \rangle.\]
        The component in Adam's degree \(j\) is \(A^!_{j-i}\otimes_k M_i\),
        which occurs as the degree \((i,j)\) component of \(F(M)\). Moreover,
        the differentials in both complexes coincide, hence we are done.
    \end{proof}
\end{cor}

\subsection{Descending to triangulated categories}

\cref{thm-eisenbud-res} shows that the functors \(F\) and \(G\) preserve
cohomology, so it is reasonable to ask whether they descend to the
(triangulated) homotopy and derived categories which are the natural setting for
formulating statements about cohomology. We show that the answer is positive
for homotopy categories.

\begin{lemma}
    The functors \(F\) and \(G\) take cones to cones-- in other words, if
    \(f:\mathbf{M}\rightarrow\mathbf{N}\) is a morphism in \(\Ch(A\grMod)\)
    then \(F(\mathsf{cone}\,f)=\mathsf{cone}\,F(f)\), and likewise for \(G\).
    \begin{proof} 
        An easy explicit check from definition of \(F\) and \(G\), ommited for
        brevity.
    \end{proof}
\end{lemma}

\begin{theorem}[\cite{eisenbud_sheaf_2003}]
    The functors \(F\) and \(G\) descend to adjoint functors
    \[ \kom(A\grMod) \xleftrightarrows[\bar F]{\bar G} \kom(A^!\grMod)\]
    between the triangulated homotopy categories of chain complexes.
    \begin{proof}
        In a category of chain complexes, a morphism \(f:M\rightarrow N\) is
        nullhomotopic if and only the inclusion \(0\rightarrow N\rightarrow
        \mathsf{cone}\,f\) is split. Now we know \(F\) and \(G\) take cones to
        cones, and being additive functors they take split morphisms to split
        morphisms. Thus \(F\) and \(G\) factor through the homotopy categories.
        
        To see that the induced functors are functors between triangulated
        categories, we show that \(F\) and \(G\) are exact functors hence
        preserve distinguished triangles. But this can be checked at the level
        of \(\mathbb{Z}^2\)-graded modules, and is immediate since the functors
        \(F\) and \(G\) are both given by tensor product with some \(k\)-vector
        space.

        The adjunction between \(\bar F\) and \(\bar G\) follows immediately
        from the adjunction in \cref{adjunction} which identifies
        subgroups of nullhomotopic morphisms.
    \end{proof}
\end{theorem}

It is clear that this is not an equivalence, the natural maps
\(G(F(\mathbf{M}))\rightarrow \mathbf{M}\) and \(\mathbf{N}\rightarrow
F(G(\mathbf{N}))\) are not invertible in the homotopy categories.
\cref{thm-eisenbud-res} however does show that they are quasi-isomorphisms, so
one might expect \({F}\) and \({G}\) to induce isomorphisms of derived
categories. There are multiple examples throughout the literature which show
this fails-- for instance \cite{keller_koszul_2003} shows that the
complex \(A\in \deri(A\grMod)\) is a compact object (i.e.\ the functor
\(\Hom_{\deri(A\grMod)}(A,-)\) commutes with infinite direct sums) but the
object \(F(A)\cong k\langle n+1 \rangle[-n-1]\in \deri(A^!\grMod)\) is not
compact. Below we exhibit explicitly the failure of our functors to descend to
the derived category.

\begin{example}[\(G\) does not preserve quasi-isomorphisms] Let \(n=0\), so that
    \(A=k[x]\) and \({A^!=k[\xi]/(\xi^2)}\). Consider the exact complex of
    graded \(A^!\)-modules 
    \[\cdots \rightarrow A^!\langle 2 \rangle \xrightarrow{\;\xi\;} A^!\langle 1
    \rangle \xrightarrow{\;\xi\;}  A^!\xrightarrow{\;\xi\;} A^!\langle -1 \rangle
    \xrightarrow{\;\xi\;} A^!\langle -2 \rangle \rightarrow \cdots\]
    which is isomorphic to the zero complex in \(\deri(A^!\grMod)\). The functor
    \(G\) maps this to 
    \[\cdots \rightarrow 0 \rightarrow \bigoplus_q A\langle -q \rangle
    \xrightarrow{\;1+x\;} \bigoplus_q A\langle -q \rangle \rightarrow 0
    \rightarrow \cdots, \]
    which is not acyclic (the only non-zero differential is not surjective),
    hence non-zero in \(\deri(A\grMod)\).  
\end{example}

\paragraph{Bernstein-Gel'fand-Gel'fand equivalence} To work around this apparent
problem, \citeasnoun{bernstein_algebraic_1978} restricts to bounded complexes so
that a simple spectral sequence argument shows \(F\) and \(G\) preserve
acyclicity. This gives well-defined functors between the bounded derived
categories which form an adjoint equivalence-- the so-called `BGG correspondence'.

\begin{lemma}
    If \(\mathbf{M}\) is a bounded acyclic complex of finitely generated
    \(A\)-modules, then the complex \(F(\mathbf{M})\) is acyclic. Likewise, if
    \(\mathbf{N}\) is a bounded acyclic complex of finitely generated
    \(A^!\)-modules, then the complex \(G(\mathbf{M})\) is acyclic.
    \begin{proof}
        Given such an \(\mathbf{M}\), the double complex \eqref{FM-totcomp} has
        exact columns.  Then the first page of the spectral sequence (starting
        with vertical cohomology) vanishes everywhere. Since \(M^p_\bullet = 0\)
        for large \(p\), the double complex is bounded and the convergence
        theorem holds, indicating the total complex is acyclic.
         
        The argument for \(G\) is similar.
    \end{proof}
\end{lemma}

Then by Example 10.5.5\todo{prove} in \cite{weibel_introduction_2003}, \(F\) and
\(G\) descend to functors between derived categories 
\[{F_\deri: \deri^b(A\grMod)\rightarrow \deri(A^!\grMod)}, \qquad G_\deri:
\deri^b(A^!\grMod) \rightarrow \deri(A\grMod).\] 
To conclude, we show that \(F_\deri\) in fact has image \(\deri^b(A^!\grMod)\)
and likewise for \(G_\deri\).

\begin{lemma}\label{derived-image-bounded}
    If \(\mathbf{M}\) is a bounded complex of finitely
    generated \(A\)-modules, then the complex \(F(\mathbf{M})\) has bounded
    cohomology and is quasi-isomorphic to a bounded complex of finitely
    generated \(A^!\)-modules. 

    Likewise, if \(\mathbf{N}\) is a bounded complex of finitely
    generated \(A^!\)-modules, then the complex \(G(\mathbf{N})\) has bounded
    cohomology and is quasi-isomorphic to a bounded complex of finitely
    generated \(A^!\)-modules.
    \begin{proof}
        If \(\mathbf{N}\) is as given, then the complex \(G(\mathbf{N})\) is
        bounded by definition-- for any \(p\), we have that the module
        \(N^p_\bullet\) is finitely generated hence has only finitely many
        graded components. Then for sufficiently large \(i\), we have
        \(N^p_{p-i}=0\) for all \(p\). 

        For \(\mathbf{M}\) as given the double complex \eqref{FM-totcomp} which
        computes \(M\) is bounded, and by \cref{regularity-criterion} the first
        page of the corresponding spectral sequence (starting with horizontal
        cohomology) has finite support. Thus by the convergence theorem for
        spectral sequences, the cohomology of \(F(\mathbf{M})\) is bounded. The
        existence of the quasi-isomorphic bounded complex of finitely generated
        modules follows from \citeasnoun{hartshorne_algebraic_2008}, III
        Lemma 12.3.
    \end{proof}
\end{lemma}

\begin{theorem}[\citeasnoun{bernstein_algebraic_1978}]
    The functors \(F\) and \(G\) induce an equivalence of derived categories 
    \[ \deri^b(A\grMod) \xleftrightarrows[F_\deri]{G_\deri} \deri^b(A^!\grMod).\]
    \begin{proof}
        From \cref{derived-image-bounded}, the functors given are well-defined.
        Then \cref{thm-eisenbud-res} shows that \({F_\deri \circ G_\deri}\) and
        \(G_\deri \circ F_\deri\) are naturally equivalent to the identity
        morphism, hence we have an equivalence of categories.
    \end{proof}
\end{theorem}


\section{Coherent sheaves on \Pn}\label{sec-cohPn}

\subsection{The Tate resolution and Beilinson monads}

\section{Koszul duality after Keller (2003)}\label{sec-keller}

\bibliography{references}
\addcontentsline{toc}{section}{References}

\end{document}
