\documentclass[a4paper]{article}
\usepackage[hmargin={30mm,30mm},vmargin={30mm,30mm}]{geometry}


% FONT
\usepackage[T1]{fontenc}
\usepackage{charter}
%\usepackage{sectsty}
%\allsectionsfont{\scshape}


% CITATION STYLE
\usepackage{harvard}


% MATHS
\usepackage{euler}
\usepackage{amsmath, amsthm, amssymb, mathtools, stmaryrd}
\usepackage{tikz-cd}
\usepackage{enumerate}
%\usepackage{unicode-math}


% LINE SPACING
\usepackage[parfill]{parskip}
\usepackage[capitalise]{cleveref}
\renewcommand{\baselinestretch}{1.25}


% TODOS
\usepackage[colorinlistoftodos]{todonotes}


% TABLE OF CONTENTS
\usepackage{tocloft}
\renewcommand\contentsname{}
\renewcommand\cftsecfont{}
\renewcommand\cftsecpagefont{}
\setlength\cftbeforesecskip{3pt}
\setcounter{tocdepth}{2}


% THEOREM STYLES
\theoremstyle{definition}
\newtheorem{defn}{Definition}[section]

\newtheorem{theorem}[defn]{Theorem}
\newtheorem*{theorem*}{Theorem}
\newtheorem{prop}[defn]{Proposition}
\newtheorem{lemma}[defn]{Lemma}
\newtheorem{cor}[defn]{Corollary}

\newtheorem{example}[defn]{Example}

\theoremstyle{remark}
\newtheorem{remark}[defn]{Remark}


% NEW ARROWS
\usepackage{stackrel}
\newcommand{\leftrarrows}{\mathrel{\raise.75ex\hbox{\oalign{%
  $\scriptstyle\leftarrow$\cr
  \vrule width0pt height.5ex$\hfil\scriptstyle\relbar$\cr}}}}
\newcommand{\lrightarrows}{\mathrel{\raise.75ex\hbox{\oalign{%
  $\scriptstyle\relbar$\hfil\cr
  $\scriptstyle\vrule width0pt height.5ex\smash\rightarrow$\cr}}}}
\newcommand{\Rrelbar}{\mathrel{\raise.75ex\hbox{\oalign{%
  $\scriptstyle\relbar$\cr
  \vrule width0pt height.5ex$\scriptstyle\relbar$}}}}
\newcommand{\longleftrightarrows}{\leftrarrows\joinrel\Rrelbar\joinrel\lrightarrows}

\makeatletter
\def\leftrightarrowsfill@{\arrowfill@\leftrarrows\Rrelbar\lrightarrows}
\newcommand{\xleftrightarrows}[2][]{\ext@arrow 3399\leftrightarrowsfill@{#1}{#2}}
\makeatother

\usepackage{scalerel}
\newcommand{\simrightarrow}{\mathrel{\ooalign{
     $\to$\cr
     \hidewidth\raise.3em\hbox{$\scaleobj{.7}{\sim}\mkern7mu$}\cr
    }
  }
}


% CUSTOM COMMANDS
\newcommand{\grMod}{\ensuremath{\text{-grMod}}}
\newcommand{\Mod}{\ensuremath{\text{-Mod}}}
\newcommand{\Sym}{\ensuremath{\text{Sym}\,}}
\newcommand{\exterior}{\ensuremath{\Lambda^\bullet\,}}

\newcommand{\Ch}{\ensuremath{\text{Ch}}}
\newcommand{\coker}{\ensuremath{\text{coker}}}
\newcommand{\img}{\ensuremath{\text{im}}}
\newcommand{\Hom}{\ensuremath{\text{Hom}}}

\newcommand{\Proj}{\ensuremath{\text{Proj}}}
\newcommand{\Pn}{\ensuremath{\mathbb{P}^n}}
\newcommand{\coh}{\ensuremath{\text{coh-}}}

\newcommand{\gnab}{{\textexclamdown}}

% DOCUMENT
\title{(Co)Derived Equivalences in Algebra and Geometry}
\date{\vspace{-2.5em} Easter 2022}

\begin{document} 
\bibliographystyle{agsm}

\maketitle
\tableofcontents 

\vspace{1em}

Fix a field \(k\), and let \(X\) and \(\Xi\) be dual \(k\)-vector spaces of
dimension \(n+1\) with dual bases \((x_i)\) and \((\xi_i)\) respectively. The goal
of this exposition is to examine equivalences of various categories that arise
naturally in this setting from algebro-geometric constructions. In particular,
we look at chain complexes of 
\begin{enumerate}[(i)]
    \item modules over the symmetric algebra \(A:=
        \Sym^\bullet(V)\),
    \item modules over the exterior algebra \(A^!:= \exterior(V)\),
    \item coherent sheaves over \(\Pn := \Proj(\Sym^\bullet (V))\), the
        projectivisation of \(\Xi\).
\end{enumerate}

\begin{example}\label{setstage}
In the simplest case when \(X, \Xi\) are one-dimensional, the data of a module
over \(A = k[x]\) involves a \(k\)-vector space \(M\) with a map
\(M\xrightarrow{x} M\) which can be seen as a cochain complex \(F(M)\) of
\(k\)-vector spaces with differential \(d\) of degree \(1\). The graded algebra
\(A^! = k[\xi]/(\xi^2)\) acts naturally on \(F(M)\) via the degree \(1\)
chain map \(F(M)\xrightarrow{\xi} F(M)\) given by
\[\begin{tikzcd}
    \cdots \arrow[r] & 0 \arrow[r] \arrow[rd] & M \arrow[r] \arrow[rd, "1"] & M
    \arrow[r] \arrow[rd] & \cdots &   \\ & \cdots \arrow[r] & M \arrow[r] & M
    \arrow[r] & 0 \arrow[r] & \cdots
\end{tikzcd}.\]
Consider the cochain complex of \(A\)-modules
\[\cdots \rightarrow 0 \rightarrow M\otimes_k A
\xrightarrow{ d\otimes 1 + \xi \otimes x} M\otimes_k A \rightarrow 0 \rightarrow
\cdots ,\]
concentrated in degrees \(-1\) and \(0\). This has the same underlying vector
spaces as the complex \(F(M)\otimes_k A\), but the differential has been
`twisted' to remember the \(A^!\)-action. This complex is exact everywhere
except in degree \(0\), where it has cohomology \(M\). Since the modules
appearing in it are free, we have recovered a free resolution of \(M\). 
\end{example}

This is the first example of what may be called \textit{Koszul duality}, a broad
term encompassing various equivalences across algebra, geometry, and representation
theory. The duality between symmetric and exterior algebras over finite
dimensional vector spaces was first studied by
\citeasnoun{bernstein_algebraic_1978}, who exhibit an adjunction between the
categories of cochain complexes of graded modules over \(A\) and \(A^!\).  
\begin{theorem*}
    There are adjoint functors
    \[ \Ch(A\grMod) \xleftrightarrows[F]{G} \Ch(A^!\grMod)\]
    such that any complex \(M^\bullet\) of graded
    \(A\)-modules has free resolution \(GF(M^\bullet)\), and any complex
    \(N^\bullet\) of graded \(A^!\)-modules has injective resolution
    \(FG(N^\bullet)\). 
\end{theorem*}
In \cref{sec-BGG}, we look at \possessivecite{eisenbud_sheaf_2003}
treatment of the Bernstein-Gel'fand-Gel'fand (BGG) correspondence described
above. To turn the adjunction into an equivalence of categories, we need to
employ the machinery of Verdier's \textit{derived categories}. Passing to the
corresponding derived categories of modules has the effect that all
\textit{quasi-isomorphisms} (i.e.\ chain maps that induce isomorphisms on
homology) become isomorphisms.

\citeasnoun{bernstein_algebraic_1978} use the correspondence between coherent
sheaves on \Pn\ and graded \(A\)-modules (see for example Chapter II of
\citeasnoun{hartshorne_algebraic_2008}) to describe the derived category
\(D^b(\coh\Pn)\) of projective \(n\)-space. In \ref{sec-cohPn} we use
\todo{finish this paragraph} 

\todo{insert paragraph about lefevre}

\section{Categories of complexes}\label{sec-trianglecat}

\subsection{Chain complexes}

\begin{example}[Modules over an algebra]
\end{example}

\begin{example}[Comodules over a coalgebra]
\end{example}

\section{The Bernstein-Gel'fand-Gel'fand correspondence}\label{sec-BGG}

\todo{describe section}

\subsection{Data}

\subsubsection{Symmetric and exterior algebras}

Given an \(n+1\)-dimensional \(k\)-vector space \(V\),
the \textit{tensor algebra} is the \(k\)-vector space 
\[ T(V) = \bigoplus_{i\geq 0}V^{\otimes i}\] 
with a product \(\nabla: T(V)\otimes T(V) \rightarrow T(V)\) induced by the
natural identifications \(V^{\otimes i}\otimes V^{\otimes j} \simrightarrow
V^{\otimes(i+j)}\). This is an associative algebra with a natural
\(\mathbb{Z}_{\geq 0}\)-grading. The \textit{symmetric algebra}
\(\Sym^\bullet(V)\) and the \textit{exterior algebra} \(\exterior(V)\) are then
the graded algebras defined as quotients of \(T(V)\) by certain two-sided
ideals, namely
\[\Sym^\bullet(V) = \frac{T(V)}{(x\otimes y - y\otimes x \;|\; x,y\in V)},
\qquad \exterior(V) = \frac{T(V)}{(x\otimes x \;|\; x\in V)}.\]
Since the ideals are generated by homogeneous elements, these algebras
inherit gradings from \(T(V)\).

We continue to use \(\nabla\) for the product morphism on either algebra,
though the corresponding bilinear map on \(\exterior{V}\) is often written
\(\wedge\).

Since we are primarily concerned with the algebras \(A=\Sym^\bullet(X)\) and
\(A^!=\exterior(\Xi)\), we redefine the grading on \(A^!\) as
\(A^!_{-i}=\Lambda^i \Xi\). This amounts to a change of sign from the usual
grading, but the convention ensures that the dual vector spaces \(X\) and
\(\Xi\) lie in degrees \(1\) and \(-1\) in their respective algebras.

\subsubsection{The exterior coalgebra}
\label{subsubsec-exteriorcoalgebra}

The \textit{exterior coalgebra} on \(\Xi\) is defined as the linear dual of
\(A^!\), written \(A^\gnab := \Hom_k(A^!, k)\). \(A^\gnab\) has the
\(\mathbb{Z}\)-grading \(A^\gnab_i=\Hom_k(A^!_{-i}, k)\) and is naturally an
\(A^!\)-module via \(a \cdot f(a') = (-1)^{\text{deg}\, a} f(a\wedge a')\) for
\(a\in A^!\) homogeneous, \(f\in \Hom(A^!,k)\). Moreover, for any \(k\)-vector
space \(N\) we have the natural isomorphism of \(A^!\)-modules \(\Hom_{k}(A^!,
N)\cong A^\gnab \otimes_k N\).  

Choosing a basis \(x_i\) for \(X\) fixes an isomorphism \(X\cong\Hom_k(\Xi,
k)=A^\gnab_1\), which can be extended to get the isomorphism of graded \(k\)-vector
spaces
\[A^\gnab = \bigoplus_i \Hom_k(\Lambda^i\Xi, k) \cong \bigoplus_i \Lambda^i X =
\exterior(X).\] 
Write \(\tau:A^\gnab \rightarrow A\) for the \(k\)-linear map which identifies
the subspaces of \(A^\gnab\) and \(A\) corresponding to \(X\), and is \(0\)
elsewhere. 


\paragraph{The coproduct on \(A^\gnab\).}
Being the linear dual of a finite dimensional algebra, \(A^\gnab\) has a natural
(coassociative counital) coalgebra structure which comes from dualising the
(associative unital) product \({\nabla: A^!\otimes_k A^!\rightarrow A^!}\). This
is called the \textit{shuffle coproduct}, and we give an explicit description of
it as follows. Given a collection of indices \(\underline{\alpha}=\{\alpha_1\,<\,
...\,<\,\alpha_i\}\subseteq \{0,...,n\}\), write \(x_{\underline{\alpha}}\) for the
standard basis element of \(A^\gnab\) given by \({x_{\alpha_1}\wedge
x_{\alpha_2} \wedge ... \wedge x_{\alpha_i}}\) (in particular,
\(x_\emptyset = 1\)). The vector \(\xi_{\underline{\alpha}}\) is defined
similarly. We say a tuple \((\underline{\beta}, \underline{\beta'})\) of subsets
is a \textit{break} of \(\underline{\alpha}\) if \((\beta_1\,<...<\,\beta_p,
\beta'_1\,<...<\,\beta'_q)\) is a permutation of \((\alpha_1\,<...<\,
\alpha_i)\) (in other words, \(\underline{\alpha} = \underline{\beta} \sqcup
\underline{\beta'}\)). The \textit{sign} of this break, written \(\langle
\underline{\beta}, \underline{\beta'}\rangle\), is defined to be the sign of the
corresponding permutation. Thus we have have 
\[\nabla(x_{\underline{\beta}} \otimes
    x_{\underline{\beta'}})=x_{\underline{\beta}} \wedge
    x_{\underline{\beta'}}=\langle \underline{\beta}, \underline{\beta'}\rangle
\, x_{\underline{\alpha}}.\] 
This allows us to write the coproduct on \(A^\gnab\) as 
\[\Delta(x_{\underline{\alpha}}) =
    \smashoperator[r]{\sum\limits_{(\underline{\beta},\underline{\beta'})\in
    \text{br}(\underline{\alpha})}} \langle \underline{\beta},
\underline{\beta'}\rangle \, x_{\underline{\beta}} \otimes
x_{\underline{\beta'}}\]
where \(\text{br}(\underline{\alpha})\) is the set of all breaks of
\(\underline{\alpha}\).  Recalling that \(A^\gnab\otimes_k A^\gnab\) is
\(\mathbb{Z}\)-graded with \(\bigoplus_{p+q=i}A^\gnab_p\otimes A^\gnab_q\) in
degree \(i\), we observe that the map \(\Delta\) respects grading hence
\(A^\gnab\) is a \textit{graded coalgebra}.

\subsubsection{Graded chain complexes}
\label{subsec-chaincomp}

Objects of \(\Ch(A\grMod)\) are chain complexes of graded \(A\)-modules in
which the differentials are morphisms in \(A\grMod\) (i.e.\ \(A\)-module
homomorphisms which preserve degree). Such an object can be viewed as a
\(\mathbb{Z}^2\)-graded \(k\)-vector space \(\bigoplus_{i,j}M^i_j\) with an
endomorphism \(d\) (the differential) such that 
\begin{enumerate}[]
    \item \(d\circ d=0\), 
    \item \(d\) has degree \((1,0)\) i.e.\ \(d(M^i_j)\subseteq M^{i+1}_j\),
        and
    \item for each \(i\in \mathbb{Z}\), \(M^i_\bullet = \bigoplus_{j}
        M^i_j\) is a graded \(A\)-module.  
\end{enumerate} 
Likewise, an object \(N\in \Ch(A^!\grMod)\) can be seen as a \(\mathbb{Z}^2\)-graded
\(k\)-vector space \(\oplus_{i,j}N^i_j\) with a differential \(\partial\) of
degree \((1,0)\). We shall use the two viewpoints on interchangeably, switching
between them whenever convenient to provide a clearer picture. In particular,
the ability to view a complex as a single module with additional structure
allows for cleaner definitions and proofs, see for instance \cref{adjunction}.

For a chain complex \(\mathbf{M}=\bigoplus_{i,j}M^i_j\), we say the lower
indices denote the \textit{internal} (or \textit{Adam's}) grading, while the
upper indices denote the \textit{differential} (or \textit{cohomological})
degree.  We use `\(\langle\cdot\rangle\)' to denote shifts in Adam's gradings,
continuing to use `\([\cdot]\)' to denote shifts in differential gradings.  Thus
for example we have \(M\langle q \rangle^i_j = M^i_{q+j}\).

\subsection{Twisted Functors}

We now define additive functors 
\[\Ch(A\grMod) \xleftrightarrows[F]{G} \Ch(A^!\grMod)\]
on which the BGG correspondence is based. In the framework of
\(\mathbb{Z}^2\)-graded vector spaces described in \cref{subsec-chaincomp}, we have 
\[\bigoplus_{i,j}F(\mathbf{M})^i_j \cong \Hom_{k}\left(A^!,\,
\bigoplus_{p,q}M^p_q\right) = A^\gnab \otimes_k
\left(\bigoplus_{p,q}M^p_q\right), \qquad \bigoplus_{i,j}G(\mathbf{N})^i_j \cong
A\otimes_k \left(\bigoplus_{p,q}N^p_q\right).\] 
However, care is needed to define the gradings and differentials since, for
example, na\"ively applying the functor \(\Hom_k(A^!,-)\) would lose all
\(A\)-module structure. The key is to modify the na\"ive differential by adding a
`twist' as in \cref{setstage}.

\subsubsection{Defining the functor \(F\)} 
We first define \(F\) on the category \(A\grMod\), seen as the full subcategory of
\(\Ch(A\grMod)\) whose objects are complexes concentrated in differential degree
\(0\). If \(M^0_\bullet\) is a graded \(A\)-module, we define \(F(M^0_\bullet)\)
to be the chain complex of \(A^!\)-modules given by a
\begin{align*} 
    \cdots \rightarrow A^\gnab\langle -i \rangle \otimes_k M^0_i
    &\xrightarrow{\;\partial\;} A^\gnab \langle -i-1 \rangle \otimes_k
    M^0_{i+1} \rightarrow \cdots \\ 
    a\otimes m &\longmapsto \sum_\alpha \xi_\alpha a \otimes x_\alpha m.  
\end{align*} 
The module \(A^\gnab\langle -i \rangle \otimes_k M^0_i\) is naturally
isomorphic to \(\Hom_k(A^!\langle i \rangle, M^0_i)\) and inherits an Adam's
grading from \(A^\gnab\) with the vector space \(A^\gnab_{j-i}\otimes_k M^0_i\)
forming the \(j\)th graded piece. These shifts in grading have been chosen
precisely so that the differential \(\partial\) has degree \((1,0)\),  while
the graded commutativity of \(A^!\) implies \(\partial\circ \partial=0\).
Thus we indeed have a chain complex of \(A^!\)-modules. 

Given a morphism \(M^0_\bullet \rightarrow M^1_\bullet\) in \(A\grMod\),
the functoriality of tensor products induces \(A^!\)-module
homomorphisms \(A^\gnab\langle-i \rangle \otimes_k M^0_i \rightarrow
A^\gnab\langle-i\rangle \otimes_k M^0_i\) which are compatible with the
differentials (i.e.\ the natural squares commute). Thus we have an additive functor
\({F:A\grMod\rightarrow \Ch(A^!\grMod)}\).

To extend \(F\) to arbitrary chain complexes \(\mathbf{M}=(\bigoplus_{i,j}M^i_j,
d)\in \Ch(A\grMod)\), we observe that the functoriality of \(F\) gives us a (commuting) bicomplex
\[\begin{tikzcd}[row sep=large]
    \vdots 
           &            
           & 
           & \vdots 
           & \vdots 
           & \\
    F(M^{i+1}_\bullet) \arrow[u]
           & 
           & \cdots \arrow[r]
           & A^\gnab \langle -j \rangle\otimes_k M^{i+1}_j \arrow[r] \arrow[u]
           & A^\gnab \langle -j-1 \rangle\otimes_k M^{i+1}_{j+1} \arrow[r] \arrow[u]
           & \cdots \\
    F(M^{i}_\bullet) \arrow[u]
           & 
           & \cdots \arrow[r]
           & A^\gnab \langle -j \rangle\otimes_k M^{i}_j  \arrow[r] \arrow[u]
           & A^\gnab\langle -j-1 \rangle\otimes_k M^{i}_{j+1}  \arrow[r] \arrow[u]
           & \cdots \\
    \vdots \arrow[u]
           &            
           & 
           & \vdots \arrow[u] 
           & \vdots \arrow[u]
           &   
\end{tikzcd}\]
where the functorially induced vertical maps are \(1\otimes d\). Define
\(F(\mathbf{M})\) to be the total complex of this bicomplex, given by
\begin{gather*}
    \cdots \rightarrow \bigoplus_{p+q=i}A^\gnab\langle -q \rangle \otimes_k
    M^p_q \xrightarrow{\;\partial\;} \bigoplus_{p+q=i+1}A^\gnab\langle -q
    \rangle\otimes_k M^p_q \rightarrow \cdots,  \\ \\
    \partial: a\otimes m \longmapsto a\otimes dm + (-1)^{\#m}\sum_\alpha
    \xi_\alpha a \otimes x_\alpha m
\end{gather*}
where \(\#m\) is the differential degree of \(m\in \mathbf{M}\). It is clear
that each \(F(\textbf{M})^i_\bullet = \bigoplus_{p+q=i}A^\gnab\langle -q\rangle
\otimes_k M^p_q\) is a graded \(A^!\) module, and the signs introduced in the
total complex construction ensure \(\partial \circ \partial = 0\). An explicit
check confirms \(\partial\) has degree \((1,0)\), so we indeed have an object of
\(\Ch(A^!\grMod)\).

\paragraph{The twist using comodules.}
\label{comoduletwist}
Observe that the differential \(\partial\) differs from the na\"ive differential
\(1\otimes d\) on the tensor product by the horizontal maps, which are the
`twists' we have been alluding to. These have a nice description using the fact
that a graded module \(N_\bullet\in A^!\grMod\) has the structure of a
graded \(A^\gnab\)-comodule via the map
\begin{align*}
    \Delta: \quad N_\bullet &\longrightarrow N_\bullet \otimes_k A^\gnab \\
    n &\longmapsto \smashoperator[r]{\sum\limits_{\underline{\alpha}\subseteq
    \{0,...,n\}}} \xi_{\underline{\alpha}} n
    \; \otimes   x_{\underline{\alpha}}.
\end{align*}

Applying this idea to the \(A^!\)-modules \(A^\gnab\langle -i \rangle\), we get
a commuting square
\[\begin{tikzcd}[column sep=large, row sep=huge]  
    \bigoplus\limits_{u+v=j-q}A^\gnab_{u} \otimes_k A^\gnab_v \otimes_k
    M^{i-q}_q 
    \arrow[r, "1\otimes \tau \otimes 1"]
    & A^\gnab_{j-q-1} \otimes_k A_1 \otimes_k M^{i-q}_{q} 
    \arrow[d, "1\otimes \nabla"] \\
    A^\gnab_{j-q} \otimes_k M^{i-q}_q
    \arrow[u, "\Delta \otimes 1"] 
    \arrow[r, "(-1)^{i-q}(\partial \,- \,1\otimes d)"]
    &A^\gnab_{j-q-1} \otimes_k M^{i-q}_{q+1}
\end{tikzcd}\]

where \(\nabla: A\otimes_k M^{i-q}_\bullet \rightarrow M^{i-q}_\bullet\) defines the
\(A\)-module structure on \(M\), and \(\tau:A^\gnab \rightarrow A^!\) is the
morphism defined in \cref{subsubsec-exteriorcoalgebra} which identifies
\(A^\gnab_1\) with \(A_{1}\), annihilating other graded pieces. 

In summary, \(F(\mathbf{M})\) as a \(\mathbb{Z}^2\)-graded vector space is
simply \(A^\gnab \otimes_{k} \mathbf{M}\)  with \((i,j)\)th piece
\[F(\mathbf{M})^i_j = \bigoplus_{p+q=i}A^\gnab_{j-q}\otimes_k M^p_q\]
and differential given on \(A^\gnab_{j-q}\otimes_k M^p_q\) by 
\[1\otimes d + (-1)^{p}(1\otimes \nabla)\circ (1\otimes \tau \otimes
1)\circ(\Delta\otimes 1).\]

\subsubsection{The left adjoint to \(F\)}

The functor \(G:\Ch(A^!\grMod)\rightarrow \Ch(A\grMod)\) is analogously defined,
and maps the chain complex \(\mathbf{N}=(\bigoplus_{i,j}N, \partial)\) to \(G(\mathbf{N})\) given by
\begin{gather*}
    \cdots \rightarrow \bigoplus_{p-q=i} N^p_q \otimes_k A\langle -q \rangle
    \xrightarrow{\;d\;} \bigoplus_{p-q=i+1}N^p_q \otimes_k A\langle -q \rangle
    \rightarrow \cdots \\ \\
    d: n\otimes a \longmapsto \partial n\otimes a + (-1)^{\#n} \sum_\alpha
    \xi_\alpha n \otimes x_\alpha a
\end{gather*}
where \(\#n\) is the differential degree of \(n\in \mathbf{N}\). The Adam's
grading on each \(G(\mathbf{N})^i_\bullet\) is inherited from \(A\), and is
given by
\[G(\mathbf{N})^i_j =   \bigoplus_{p-q=i}N^p_q \; \otimes_k A_{j-q}.\]
Recalling that every \(A^!\)-module is a \(A^\gnab\)-comodule (see
\cref{comoduletwist}), we can use the comodule structure-map \(\Delta: N^i_\bullet
\rightarrow N^i_\bullet \otimes A^\gnab \) to define the differential on
\(N^p_q\otimes_k A_{j-q}\) as
\[\partial \otimes 1 + (-1)^{p}(1\otimes \nabla)\circ (1\otimes \tau \otimes
1)\circ(\Delta \otimes 1).\]

\paragraph{The adjunction.} 
Having defined the functors \(F\) and \(G\), we show that \(G\) is left adjoint
to \(F\). Spelled out this means given \(\mathbf{M}\in \Ch(A\grMod)\) and
\(\mathbf{N}\in \Ch(A^!\grMod)\), there is a natural isomorphism of abelian
groups
\[\Hom_{\Ch(A\grMod)}(G(\mathbf{N}), \mathbf{M}) \cong
\Hom_{\Ch(A^!\grMod)}(\mathbf{N}, F(\mathbf{M})).\]
At its heart this is a \(\otimes\)-\(\Hom\) adjunction, as we shall illustrate in
the special case of module categories below. 

\begin{lemma}[\citeasnoun{eisenbud_sheaf_2003}] \label{adjunction-simple}
    Given modules \(M\in A\Mod\) and \(N\in A^!\Mod\), there are natural
    isomorphisms of abelian groups
    \[\Hom_{A}(A\otimes_k N, M) \cong \Hom_k(N,M) \cong \Hom_{A^!}(N,
    \Hom_{k}(A^!,M)).\]
    \begin{proof}
        Choosing a basis \(n_\alpha\) for \(N\), the first isomorphism follows
        from observing that the \(A\)-module \(A\otimes_k N\) is freely
        generated by \(1\otimes n_\alpha\). 

        The second isomorphism sends \(\varphi\in \Hom_k(N,M)\) to the map
        \(\varphi^!: N\rightarrow \Hom_k(A^!, M)\) such that for any \(n\in N\)
        and \(a\in A^!\) homogeneous we have 
        \[\varphi^!(n)(a) = (-1)^{\text{deg}\,a}\varphi(an)\] 
        The inverse correspondence sends \(\varphi^!\in \Hom_{A^!}(N,
        \Hom_k(A^!,M))\) to \(\varphi\in \Hom_k(N,M)\) given by 
        \[\varphi(n)= \varphi^!(n)(1).\]
    \end{proof}
\end{lemma}

We now exhibit the general adjunction for \(F\) and \(G\), and it is here that
the flexibility of interpreting a chain complex \(\mathbf{M}\) of graded modules
as a single \(\mathbb{Z}^2\)-graded module \(\bigoplus_{i,j}M^i_j\) (see
\cref{subsec-chaincomp}) really comes handy.  Interpreting \(\Ch(A\grMod)\) as a
subcategory of \(A\Mod\) (likewise for \(A^!\)), we use \cref{adjunction-simple}
to identify \(\Hom_{\Ch(A\grMod)}(G(\mathbf{N}),\mathbf{M})\subset
\Hom_A(\mathbf{N}\otimes_k A, \mathbf{M})\) and
\(\Hom_{\Ch(A^!\grMod)}(\mathbf{N},F(\mathbf{M}))\subset \Hom_{A^!}(\mathbf{N},
\Hom_k(A^!,\mathbf{M}))\) with the same subgroup of
\(\Hom_k(\mathbf{N},\mathbf{M})\).

\begin{theorem}[\citeasnoun{bernstein_algebraic_1978}] \label{adjunction}
    The functor \(G\), from the category of complexes of graded \(A^!\)-modules
    to the category of complexes of graded \(A\)-modules, is a left adjoint to
    the functor \(F\).
    \begin{proof}
        Given \(\bar{\varphi}\in \Hom_{A}(G(\mathbf{N}),
        \mathbf{M})\), the corresponding map \(\varphi\in
        \Hom_k(\mathbf{N},\mathbf{M})\) found in \cref{adjunction-simple} is
        given by \(\varphi(n)=\bar\varphi(n\otimes 1)\). Thus \(\bar\varphi\)
        has degree \((0,0)\) if and only if \(\bar\varphi(N^i_j \otimes_k A_0)
        \subseteq M^{i-j}_j\), if and only if \(\varphi(N^i_j)\subseteq
        M^{i-j}_j\).  Moreover for \(n\in N^i_j\), direct computation shows
        \begin{align*}
            (d_M \circ \bar\varphi - \bar\varphi \circ d_{G(\mathbf{N})}) (n
            \otimes 1) \; = \; (d_M \circ \varphi - \varphi \circ
            \partial_N) (n) - (-1)^i \sum_\alpha x_\alpha \varphi(\xi_\alpha n),
        \end{align*}
        thus \(\bar\varphi\) is a morphism in \(\Ch(A\grMod)\) if and only if
        \begin{equation}\label{adjunctcondition}
            \varphi(N^i_j)\subseteq M^{i-j}_j, \quad \text{and} \quad d_M\circ
            \varphi - \varphi \circ \partial_N = \sum_\alpha x_\alpha \varphi
            \xi_\alpha        
        \end{equation}
        where we write \(\sum_\alpha x_\alpha\varphi \xi_\alpha\) for the map
        that takes \(n\in N^i_j\) to \((-1)^i\sum_\alpha x_\alpha
        \varphi(\xi_\alpha n)\).

        Likewise given \(\varphi^!\in \Hom_{A^!}(\mathbf{N},F(\mathbf{M}))\),
        repeating the above argument shows \(\varphi^!\) is an element of
        \(\Hom_{\Ch(A^!\grMod)}(\mathbf{N},F(\mathbf{M})\) if and only if
        the corresponding map \(\varphi \in \Hom_k(\mathbf{N},\mathbf{M})\)
        satisfies \eqref{adjunctcondition}. This shows that the isomorphisms
        given in \cref{adjunction-simple} restrict to isomorphisms
        \[\Hom_{\Ch(A\grMod)}(G(\mathbf{N}),\mathbf{M})\cong \{\varphi\in \Hom_k(\mathbf{N},\mathbf{M}) \text{ satisfying \eqref{adjunctcondition}}\} \cong \Hom_{\Ch(A^!\grMod)}(\mathbf{N},F(\mathbf{M})) \]
        thereby showing \(G\) is left adjoint to \(F\).
    \end{proof}
\end{theorem}

\begin{example}
\end{example}

\subsubsection{The (co)unit of adjunction}

\section{Coherent sheaves on \Pn}\label{sec-cohPn}

\subsection{The Tate resolution and Beilinson monads}

\section{Koszul duality after Keller (2003)}\label{sec-keller}

\bibliography{references}
\addcontentsline{toc}{section}{References}

\end{document}
